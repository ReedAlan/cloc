% User Documentation for the CLOC package
% Author Alan Reed
% INTERNET  A.REED@BHAM.AC.UK
% UK JANET network  A.REED@UK.AC.BHAM
% last updated 13 January 1992
% Copyright (C) Alan Reed 1984
\documentstyle{article}
%    The following instructions are for A4 paper
%
% 1 inch margins at left and right:
\oddsidemargin=0in
\evensidemargin=0in
\textwidth=6.25truein          % A4 paper is 8.25in wide

% 1 inch margins at top and bottom:
\headheight=0pt
\headsep=0pt
\topmargin=0in
\textheight=9.7in              % A4 paper is 11.7in high

\sloppy
\begin{document}
\title{CLOC mark 2Q}
\author{A Collocations and Concordance Package\\by\\
        Alan Reed\thanks{UK JANET network address A.REED@UK.AC.BHAM}}
\maketitle

\begin{quote}
\begin{center}
{\bf Abstract}
\end{center}

This guide describes a computer program which will help you analyse
natural language text.
The program is named CLOC and takes the form of a
package of facilities designed for ease of use by people with little or
no computer experience.
The original development work has been done at Birmingham University
Computer Centre in collaboration with Professor J. McH. Sinclair of the
Department of English Language and Literature.
The package has been improved from the original and currently contains
facilities for producing sorted vocabulary lists, word indexes, concordances,
and the automatic discovery of collocations.
The name CLOC is an acronym taken from the term `ColLOCation'.
\end{quote}

\begin{center}
{\bf Acknowledgements}
\end{center}
\begin{quote}
The author would like to offer his thanks to friends and
colleagues whose advice and criticism have aided the writing of this
guide and the creation of the package.
To Professor J. McH. Sinclair of the Department of English for requesting
the program and suggesting several of the features; to Dr. J. L. Schonfelder
for his enthusiasm and advice; and to Professors Greaves (York, Toronto),
Benson (York,Toronto) and Brainerd (Toronto) whose interest and support
was most welcome.
\end{quote}
\newpage

\tableofcontents
\newpage

\section{INTRODUCTION}
CLOC is a package which will enable a novice computer user to
analyse natural language text by computer.  This guide explains
what the package can do, and shows you how to instruct
it to carry out various tasks.

The package can examine the vocabulary used by an author and be
told to print it in several ways.  The vocabulary, or a selected
portion of it, can be printed in order, as in a dictionary, or
according to how frequently a word is used.

The primary purpose of the CLOC package is to produce collocations
of selected words.  These are frequently occurring patterns of
words which appear regularly within a text.  The package is capable
of discovering these patterns and will print the context of each
occurrence in a style of your choice.

CLOC can also produce a concordance of words selected from the
text, to show how an author uses the selected words. You can choose
the amount of context that is printed as well as the style of concordance.

You guide and control the actions of the package by employing
a few simple commands, which are supplied to the package by way of
statements in a command language.  The following sections explain
how you tell the package what analyses you want it to do.

\section{PREPARATION OF TEXT}
\subsection{THE CHARACTER SET}
Before you can use the CLOC package to analyse some text it must be
converted from the printed page or spoken word into a form that a computer
can read.
A printed page
could contain many differing alphabets, use several type styles,
and allow a large number of special symbols.  As CLOC can only deal
with (say)95 different characters, you must devise a consistent scheme to
convert every letter, punctuation mark, diacritic and significant
change of type style into one or more characters in this small
and restricted alphabet.

This can be achieved by dividing the set of possible characters into
several mutually exclusive categories.  Use one category of characters
to compose words, use another to separate one word from the next,
and so on.  For English, the first category could include the
alphabet \verb/A/ to \verb/Z/, while the second could include
\verb/? , ; ./ , and `space'.
In this document it will be assumed that text is prepared using the 95
printable characters of the ASCII alphabet. This includes the upper and
lower case letters together with a large variety of other symbols. For
your information the ASCII alphabet is reproduced below:-
\begin{quote}
\begin{verbatim}
!"#$%&'()*+,-./0123456789:;<=>?@
ABCDEFGHIJKLMNOPQRSTUVWXYZ[\]^_`
abcdefghijklmnopqrstuvwxyz{|}~
\end{verbatim}
\end{quote}
These will be found on most computer keyboards but the graphic symbols
printed on the keyboard may not look quite the same as the above.
CLOC commands can be written using uppercase or lowercase
letters or a mixture of the two. All the examples will show CLOC commands
in uppercase letters for clarity.
The CLOC package lets you choose which characters are `letters' and
which are not. Normally you would choose ~abcdefghijklmnopqrstuvwxyz~ as
your alphabet. Additional `letters' called `padding' and `deferred'
can be defined to cope with accents, apostrophes, breathing marks, etc.
If you are unlucky enough to prepare your text on
punched cards you could represent the phrase `Once upon a time' as follows:-
\begin{quote}
\begin{verbatim*}
=ONCE UPON A TIME
\end{verbatim*}
\end{quote}
Notice that before every capital letter you should place some symbol
(say = ) to indicate that a capital letter comes next. This can also
be used when a capital letter occurs inside a word as in `MacDonald'.
This you could represent as :
\begin{quote}
\begin{verbatim*}
=MAC=DONALD
\end{verbatim*}
\end{quote}
The CLOC package could be told that = is a special kind of letter so
it could read `=MAC=DONALD' as one word rather than the two words
`MAC' and `DONALD'.

\subsection{CAPITAL AND SMALL LETTERS}
The CLOC package is designed to work with text containing a mixture of
upper and lower case letters. Thus if you ask for a concordance of
(say) `the' you will get all the places where `the' occurs even when
`The' is given in the text. Usually the case of the letters is not
relevant, but if required you can tell CLOC that the case of letters
is significant, in which case `the' and `The' will be counted as
different words. (see the \verb/ITEMIZE USING/ command for details).

\subsection{DIACRITICAL MARKS}
Marks placed above or below letters to indicate stress can be
represented by two characters, one for the letter and one for the
particular mark.  For example you could represent the following:
\begin{quote}
clich\'{e} as \verb/clich1e/ (or on punched cards as \verb/CLICH1E/)
\end{quote}
where the symbol 1 is used to represent the acute accent.  This can
be combined with capitalization as follows:
\begin{quote}
\'{E}cosse as \verb/1Ecosse/ (or on punched cards as \verb/1=ECOSSE/)
\end{quote}
Note that each diacritical mark must be considered as either a `padding' or
a `deferred' `letter'.

\subsection{FOREIGN LANGUAGES}
When the language of the text is not in the Roman alphabet, you must
transliterate letters in the language to characters in the
computer's alphabet.  Using Greek as an example and a systematic
conversion changing $\alpha$ to \verb/A/, $\beta$ to \verb/B/ and $\gamma$
to \verb/G/ etc. we could write:-
\begin{quote}
$\pi o\lambda\epsilon\mu\iota\kappa o \sigma$ as  \verb/polemikos/
\end{quote}
where each Greek letter is replaced by a Roman letter.  When the
natural language being used contains words from another language
they should be carefully distinguished.  This can easily be achieved
by prefixing each rarely occurring foreign word with a special
character.  Thus when using English containing French words they
could be distinguished by a ~\$~ symbol.  For example:-
\begin{quote}
`In French, lard means bacon' could be coded as:
\end{quote}
\begin{quote}
\begin{verbatim*}
In French, $lard means bacon
\end{verbatim*}
\end{quote}
or on punched cards as
\begin{quote}
\begin{verbatim*}
=IN =FRENCH, $LARD MEANS BACON
\end{verbatim*}
\end{quote}

It is expensive and time-consuming to convert a large amount of text
into computer readable form. It is well worth finding out from
colleagues or published catalogues if your chosen text has already
been prepared for computer use. If you have to do the job yourself
do make sure you have coded all significant features in the text
using a consistent and preferable `standard' method. It is well
worth doing a pilot study using a few hundred lines of text to
get a feel for what the coding scheme should cover. Finally do not
forget to obtain copyright clearance for storing the text on a computer.

\subsection{TEXT REFERENCES}
Text prepared for a computer is likely to contain information
which is extra-textual such as the title, author, chapter, page number,
and so on. CLOC can be told that a text has no references whatever,
it which case to treats the text as a series of `words' separated
from each other by `spaces' , `fullstops', etc. When a concordance
or word index is requested CLOC will use the line number to act as a
reference for citations that are printed. It is much better though to
pre-code text references into the text to begin with using some well
known and simple convention such as that used in the COCOA or OCP
packages.

\subsubsection{Coding references}
This feature allows you to include in your text data the name
of the author, the section, chapter, and line number, etc. in a
manner similar to the well known COCOA and OCP packages.
Your text data could include say \verb*/<A DICKENS><P 1><L 1>/ which to you
would signify author Dickens, Page 1, Line 1.  Subsequent lines
would contain the text data for this section.  For the next page
it is sufficient for you to type \verb*/<P 2><L 1>/ as the author has
not changed.  As the package reads each line of text the line
number is increased by 1, so the line number should be reset to 1
when you start a new page.  Note that \verb*/<L 1>/ refers to the
{\em next line},
hence no text data should occur after it on the same line.  Apart
from this restriction, text references may be placed anywhere in
your text data.  It is, however, recommended that text references
be placed on lines separate from the text data itself.  Blank
lines and lines containing text references only are ignored.
This feature is only invoked when you supply an \verb/INPUT DETAILS/ command
with the \verb/REFS/ option in the specification field
(see section \ref{input}).
Suppose you had supplied \verb/REFS<>/ this would tell the package that
your text
references were enclosed by the the two characters \verb/</ and \verb/>/.

The general form of a text reference is: ~a{\em letter gap reference}b\\
where:-

{\em letter} is \verb/A/ or \verb/B/ or \verb/C/ \ldots \verb/Z/

{\em reference} is any sequence of characters

{\em gap} denotes one or more spaces\\
and ~ab~ are the two characters specified on the \verb/REFS/ab part of the
\verb/INPUT DETAILS/ command.  (See section \ref{input})

{\em Note:-}
\begin{itemize}
\item When L is used as a {\em letter}, the reference must be a number
      which will normally be 1.
\item An incorrect text reference will be ignored - you can therefore
      include a title of the form: a~{\em title}b which the package will
      ignore.
\end{itemize}

\subsubsection{Printing references}
Each line of a printed concordance can be prefixed with a
detailed text reference.  The \verb/CONCORDANCE/, \verb/COLLOCATIONS/,
\verb/CO-OCCURRENCE/, \verb/INDEX/ and \verb/WRITETEXT/
commands can contain the keyword \verb/REFS/ followed by {\em letter}
{\em number} pairs, e.g. \verb/REFS A4P2L3/.  This example will cause the
first 4 characters of the current \verb/A/ {\em reference}, the first 2
of the \verb/P/ {\em reference}, and a 3 figure line number to be placed
before every printed citation.  Each entry will be separated
from the next by one space.  The general rule is that for
every {\em letter number} pair occurring after the \verb/REFS/ keyword
the first {\em number}
characters of the current value of the {\em letter} references are
printed.

A minor variation occurs when a line number of say \verb/L2/ is asked
for, and the given citation occurs at line 100 or greater. On
the first occasion that this happens the package increases the
request by 1, and subsequent references are now considered to
have been requested by \verb/L3/.

Note that in the first instance all reference letters are
considered to contain `spaces': thus a request of say \verb/X4/ will
cause 4 spaces (+1 space) to be printed.

\subsubsection{Example}
The following verse is to be coded for storage on a computer:-
\begin{verse}
{\em The Small Celandine}\\
There is a Flower, the Lesser Celandine\\
That Shrinks, like many more, from cold and rain;\\
And, the first moment that the sun may shine,\\
Bright as the sun itself, 'tis out again!

When hailstones have been falling, swarm on swarm,\\
Or blasts the green field and the trees distress'd,\\
Oft have I seen it muffled up from harm,\\
In close self-shelter, like a Thing at rest.
\end{verse}

On a computer this could be coded as:-
\begin{quote}
\begin{verbatim*}
<  The Small Celandine>
<A Wordsworth><V 1><L 1>
There is a Flower, the Lesser Celandine
That Shrinks, like many more, from cold and rain;
And, the first moment that the sun may shine,
Bright as the sun itself, 'tis out again!
<V 2><L 1>
When hailstones have been falling, swarm on swarm,
Or blasts the green field and the trees distress'd,
Oft have I seen it muffled up from harm,
In close self-shelter, like a Thing at rest.
\end{verbatim*}
\end{quote}

\section{OVERVIEW OF THE CLOC PACKAGE}
You must supply instructions to the package to tell it how to
read and analyse the text.  These instructions are prepared
according to rules described in section \ref{rules}, and
must be supplied to the package in the following order:

\begin{enumerate}
\item (Optional) \verb/INPUT DETAILS/ command
\item Word definition commands
\item Word selection commands
\item Task selection commands, e.g. \verb/WORDLIST/, \verb/INDEX/,
      \verb/CONCORDANCE/,
      \verb/CO-OCCURRENCE/, \verb/COLLOCATIONS/ and \verb/WRITETEXT/ commands
\item \verb/FINISH/ command
\end{enumerate}

The precise order in which the individual commands should be
presented is described in the Appendix.  The layout and effect of
each command will be described later.

The \verb/INPUT DETAILS/ command can be used to inform the package
of any special characteristics in the way the text has been prepared,
such as what kind of text references are present.

{\em Word definition} commands inform the package how to interpret the text
data which is about to be read.  They define the constitution of a
{\em word} in terms of an alphabet of {\em characters}. The CLOC package
assumes that a file of text is composed of a series of distinguishable
words, clearly separated from each other by spaces, full stops, etc.

{\em Words selection} commands are used to instruct the package to choose a
suitable collection of words, termed `nodes', which will be used by the
task commands during the analysis of the text.

The \verb/WORDLIST/ command causes the package to list the selected words
sorted into a chosen order.

The \verb/INDEX/ command will produce for each word a list of references to
it's position in the original text.

The \verb/CONCORDANCE/ command is used to command the package to print a
concordance of the selected words.  The style chosen for printing the
concordance can be chosen from a menu of styles.

The \verb/CO-OCCURRENCE/ command will produce a concordance of given phrases or
series of words.

The \verb/COLLOCATIONS/ command is used to command the package to search the
context of the selected words, and to print the context of those
which possess frequently occurring neighbours.

The \verb/WRITETEXT/ command will cause the original text to be printed out,
but each line will start with a text reference.

The following example program illustrates how CLOC commands
could appear in a typical task.
\begin{tabbing}
Concordance command \=\verb/CONCORDANCE     KWIC, CITE 6 BY 6/\kill
                    \>\verb/ITEMIZE USING  CLOC/\\
Word definition     \>\verb/*LETTERS       abcdefghijklmnopqrstuvwxyz/\\
                    \>\verb/SELECT WORDS/\\
Word selection      \>\verb/*FREQUENCY     (100 TO 500)/\\
                    \>\verb/EXCLUDING/\\
                    \>\verb/*LIST OF WORDS i the but/\\
Concordance command \>\verb/CONCORDANCE    KWIC, CITE 6 BY 6/\\
\verb/FINISH/ command      \>\verb/FINISH/
\end{tabbing}

This example causes CLOC to do the following things:
\begin{itemize}
\item Read text composed of a series of words in the English alphabet
\item Select a collection of words each of which has a frequency of
      occurrence lying in the range 100 to 500 inclusive, with the
      exception of the words `i' `the' and `but'
\item Produce a Key Word In Context concordance of the chosen words,
      where each word is surrounded by 12 words of context.
\end{itemize}

\section{THE COMMAND LANGUAGE}
\label{rules}
The package needs to be told what actions to perform and in
which order they should be done.  It is the function of the `command
language' to supply this information in a clear and unambiguous
way.  Each of the following sections contains detailed information
on how to instruct the package to carry out certain actions.  The
sections are described in the order in which they should be presented
to the package, when they are phrased in terms of the `command
language'.

The package is controlled by a series of `control statements' each of
which contains a `command'.  These commands are obeyed by CLOC
in the order in which they are given.  Certain commands are
optional and can be included when you require more
facilities than those provided by default.

\subsection{The control statement conventions}
CLOC commands must be prepared in a standard
format.  A control statement is notionally divided into two distinct
`fields', each of which occupies a certain number of columns
on a line.  The fields for CLOC control statements are as
follows:-
\begin{itemize}
\item The {\em control field} occupying columns 1 to 15 inclusive.
      Its function is to inform the package of an action to be
      performed, or to present the package with further information.
\item The {\em specification field} occupying columns 16 to 80
      inclusive.  This portion of the line supplies extra
      information to the instruction in the control field.
\end{itemize}

An example of a command is
\begin{quote}
\begin{tabbing}
WORDLIST~~~~~~~~~\=ALPHA\\
\\
control field    \>specification field
\end{tabbing}
\end{quote}
This command instructs the package to sort the
words into alphabetic order, and to print them.

The command WORDLIST must be placed in columns 1 to 15 of the line.
The specification field, columns 16 to 80, of
this command contains the sorting criterion ALPHA,
meaning `alphabetic order'.

\subsection{Rules for Control Statements}
\begin{enumerate}
\item Each control field must be written within columns 1 to 15.
\item The commands must be spelt correctly.
      (Thus CONCORDENCE is not a valid command!)
\item \label{x} When columns 1 to 15 contain spaces only, the specification
      field of the line is treated as a continuation of the
      specification field of the previous command.
\item Keywords in the specification field must not contain
      spaces,nor be split when continuing them onto the next
      line.
\item Some commands contain a star symbol (*) in column 1.
      This indicates that the command is subsidiary to the
      last unstarred command.  The starred commands
      supply extra information to that supplied by the unstarred
      command upon which they are dependent.
      The star symbol is there for your convenience to remind you of
      the function of these commands, the symbol itself is optional.
\item The right parenthesis ) can be used to terminate a control field.
      The specification field is then deemed to start
      immediately following it.  Hence, you can write
      \verb/WORDLIST)ALPHA/.  This feature is intended for use when
      CLOC control statements are typed at a terminal rather than
      punched on cards.  Note that you can use a `)' in column 1
      to stand for the 15 space continuation field described above
      in part \ref{x} above.
\item All CLOC commands and keywords can be written in upper
      or lower case (or a mixture of the two). The examples in this
      guide are written in upper case for clarity.
\item The pseudo CLOC commands ~-INSERT~ ~-SEND~ ~-NOSEND~ do not take
      any continuation lines. They can therefore be placed anywhere
      in a sequence of CLOC commands.
\end{enumerate}

\subsection{The -INSERT feature}
This allows you to include a file of prewritten CLOC commands. You could,
for example, have a set of files each containing a different sequence
of word definition commands. Alternatively you could have files containing
specific lists of words which you could -INSERT when required.

\subsubsection{Examples}
\begin{quote}
\begin{verbatim}
-INSERT        WORDDEF1
SELECT WORDS
*FREQUENCY     (1 TO 100)
EXCLUDING
-INSERT        VERBS
\end{verbatim}
\end{quote}

\subsubsection{General form}
\begin{quote}
-INSERT~~~~~~~~~~{\em filename}
\end{quote}

The contents of {\em filename} are used as if they appeared in
place of the -INSERT command.

\subsubsection{Points to note}
\begin{itemize}
\item There are no continuation lines for this (pseudo) CLOC command.
      This allows {\em filename} to contain lists of words on continuation
      lines only. For example, if a file  A  contains:
      \begin{verbatim}
                     is are was were be
                     up down in out
      \end{verbatim}
we could write
      \begin{verbatim}
      *LIST OF WORDS
      -INSERT        A
                     EXTRA-WORDS-HERE
      \end{verbatim}

which would be interpreted as if you had written
      \begin{verbatim}
      *LIST OF WORDS
                     is are was were be
                     up down in out
                     EXTRA-WORDS-HERE
      \end{verbatim}
\item The syntax of {\em filename} depends on the computer that CLOC
      is implemented on.
\item Lines taken from an -INSERT file will be copied onto the
      CLOC information and diagnostic file. An indication will be
      given as to where the lines came from.
\item The contents of an -INSERT file may include other -INSERT commands.
      The depth of nesting is implementation dependent.
\end{itemize}

\subsection{The -SEND feature}
This (pseudo) CLOC command will cause subsequent CLOC commands to be
sent to the CLOC diagnostic and information file.

\subsubsection{Example and General form}
\begin{quote}
-SEND
\end{quote}

\subsection{The -NOSEND feature}
This (pseudo) CLOC command will stop the normal process of sending
CLOC commands to the CLOC diagnostic and information file.

\subsubsection{Example and General form}
\begin{quote}
-NOSEND
\end{quote}

\section{The INPUT DETAILS command (optional)}
\label{input}
This command allows you to specify the maximum width of lines
of text data; to indicate the presence of text references; to determine
which parts to skip; to define an explicit newline symbol; to include
ignorable comments; and to specify rules about line continuation.
\begin{quote}
When this command is absent INPUT~DETAILS~~WIDTH80 is assumed.
\end{quote}

\subsection{Examples}
\begin{quote}
\begin{verbatim}
INPUT DETAILS       WIDTH72
INPUT DETAILS       NEWLINE/,REFS<>
INPUT DETAILS       WIDTH128,NEWLINE/,CONTINUE+
INPUT DETAILS       COMMENT(),NEWLINE/,CONTINUE+,REFS<>,ENDHYPHEN-
\end{verbatim}
\end{quote}

\subsection{General Form}
\begin{quote}
\begin{verbatim}
INPUTDETAILS   WIDTHnumber,SKIPab,COMMENTab,
               NEWLINEa,CONTINUEa,RUNOVER,REFSab,ENDHYPHENa
\end{verbatim}
\end{quote}

\subsection{Default value}
\begin{quote}
INPUT~DETAILS~~WIDTH80
\end{quote}

\subsection{Parameters}
\begin{itemize}
\item WIDTHnumber~~~~~~~~default value  WIDTH80\\
At most {\em number} characters will be read for each line of text data.
Trailing spaces will be removed by the package. All characters which
occur after column {\em number} will be ignored.

\item SKIPab\\
When present this instruction causes the package to ignore all
characters between ~a~ and ~b~ inclusive.  This option withdraws
characters ~a~ and ~b~ from the available character set.

\item COMMENTab\\
Words which occur between the {\em pair of} characters ~aa~ and the {\em pair of}
characters ~bb~ will not appear in the word count tables, but they
will appear in the context when a citation is printed.  This option
withdraws the characters ~a~ and ~b~ from the available character set.

\item NEWLINEa\\
When present the character ~a~ represents a logical newline.  This
option allows more than one `line of text' to be placed on the same
line.  Note that ~a~ will also be inserted automatically at the end of
each line.  This option withdraws character ~a~ from the available
character set.

\item CONTINUEa\\
When CONTINUEa is present and ~a~ is found in the text, all characters
remaining on the line are ignored.  The next line is considered to
replace the ignored part, and to be on the same line.

\item RUNOVER\\
When RUNOVER is present, and the text reading position is at the
end-of-line (i.e. at the WIDTH number position), the end-of-line will
{\em not terminate} a word.  Hence the full width of line can be used to
store text, and words can run over onto the next line.

\item REFSab\\
When present the package extracts text references of the form
a{\em letter reference}b from the text file.  This option withdraws
characters ~a~ and ~b~ from the available character set.

\item ENDHYPHENa\\
Note that ~a~ is normally ~-~ . When present this feature affects the
reading of text which splits words via a hyphen (-) at the end of a line.
The character ~a~ (-) is removed from the word and it's two parts fused.
\end{itemize}

\section{WORD DEFINITION COMMANDS}
The CLOC package has been designed to read text coded according to
many differing conventions.  No matter how a text has been coded, the
package interprets it as an arbitrary series of {\em words}. The composition
of words is left up to you, but CLOC needs to know
what rules are used for constructing words. These rules embody a
strategy for extracting words from the characters in the text data.
The process of combining characters in this way is called {\em itemization}
and one must first select which itemizing strategy the package is to use;
the rules of the strategy are supplied by subsidiary (starred) commands.

\subsection{The ITEMIZE USING Command}
This command is used to select a strategy for itemizing the text.
Note that the -ISE form can also be used.

\subsubsection{Example}
\begin{quote}
ITEMIZE USING       CLOC
\end{quote}

\subsubsection{General form}
\begin{quote}
ITEMIZE USING       {\em strategy name}
\end{quote}

\subsubsection{Default}
\begin{quote}
When {\em strategy name} is absent CLOC is assumed.
\end{quote}

Several possibilities are available at present. They are:-
\begin{itemize}
\item CLOC\\
      This ensures that words which differ only by the case of their
      letters, and/or contain *PADDING letters (q.v.) are counted as the same
      word
\item CLOC UNCHANGED\\
      When you choose this strategy words will always be distinguished by
      the case of their letters and the presence of *PADDING letters.
\end{itemize}

For example, consider the sentence:-

The MacDonald Hotel is different from the Mac'donald Motel.

Assuming that the apostrophe ' has been designated a padding letter, then
when the CLOC itemizing strategy is in use the word `the' is deemed
to occur twice, as does the word `macdonald'. When a CONCORDANCE
or COLLOCATIONS task (etc) is run, they too will treat the various forms as
if they were the same word. The citations will of course look like the
original text. The effect of the CLOC itemizing strategy is that :-
\begin{tabbing}
Mac'donald~~~   \=is mapped to~~~     \=`macdonald'\kill
The             \>is mapped to        \>`the'\\
the             \>is mapped to        \>`the'\\
MacDonald       \>is mapped to        \>`macdonald'\\
Mac'donald      \>is mapped to        \>`macdonald'\\
Hotel           \>is mapped to        \>`hotel'\\
Motel           \>is mapped to        \>`motel'\\
is              \>is mapped to        \>`is'\\
different       \>is mapped to        \>`different'\\
from            \>is mapped to        \>`from'
\end{tabbing}
When CLOC UNCHANGED is used all the above words are considered distinct.

The ITEMIZE~USING~~CLOC command has a number of subsidiary commands.
These commands tell the package how to interpret the
characters it finds on the lines containing the text data.

\subsection{The *LETTERS Command}
This command is mandatory and must be the first command which follows the
ITEMIZE~USING~~CLOC command.  This informs the package of the
alphabet of characters out of which words are composed.
A {\em word} is defined to be one or more consecutive letters. Every character
which could form part of a word must be specified here.  This includes
characters used for accents, apostrophes, hyphenation, changes of type
style etc.

\subsubsection{Example}
\begin{quote}
*LETTERS~~~~~~~abcdefghijklmnopqrstuvwxzy
\end{quote}

\subsubsection{General form}
\begin{quote}
*LETTERS~~~~~~~{\em letter characters}
\end{quote}

The {\em order} in which the {\em letter characters} appear in the command
is significant.  This order determines the way in which
words will be alphabetically sorted.  In the above example, those
words beginning with `a' will precede those starting with `b', and so on.
Thus `alan' will sort before `fred' which itself precedes `freda'.
Note that this command automatically caters for upper and lower case letters.

\subsection{The *PADDING Command}
This command is optional and when present informs the package of
those {\em letter characters} which are to be ignored when words are placed
in the vocabulary table.  Usually this command will contain those
{\em letter characters} used as apostrophes or hyphenation,
but any characters specified on the above *LETTERS
command could also be used.

When the CLOC itemizing strategy is chosen
the *PADDING letters cannot appear in the vocabulary print-outs, because
they are absent from the vocabulary table.

Note that if you were to choose an itemizing strategy (e.g. CLOC UNCHANGED)
which allowed padding letters to appear in the vocabulary table, they
would be ignored when sorting took place.

\subsubsection{Example}
\begin{quote}
*PADDING~~~~~~~' -\\
Words containing the apostrophe and/or the hyphen will have them removed
before the word is stored in the vocabulary table.
\end{quote}

\subsubsection{General form}
\begin{quote}
*PADDING~~~~~~~{\em letter characters}\\
Where every character declared must be a {\em letter character} declared
on the *LETTERS command. This is a deliberate design decision to emphasize
that CLOC defines a word to be a sequence of {\em letter}s.
\end{quote}

\subsection{The *DEFERRED Command}
This command is optional and when present informs the package of
those {\em letter characters} which are to be ignored when words are sorted
alphabetically.  Usually this command will contain those
{\em letter characters} used for accents and changes
of type style, but any characters specified on the above *LETTERS
command could also be used. Words which contain *DEFERRED letters will
be counted separately.

\subsubsection{Examples}
\begin{itemize}
\item *DEFERRED~~~~~~-\\
Hyphenated words will be separately indexed.
\item *DEFERRED~~~~~~aeiou\\
Words will be sorted alphabetically ignoring vowels.
\end{itemize}

\subsubsection{General form}
\begin{quote}
*DEFERRED~~~~~~{\em letter characters}\\
Where every character declared must be a {\em letter character} declared
on the *LETTERS command. This is a deliberate design decision to emphasize
that CLOC defines a word to be a sequence of {\em letter}s, and that the
deferred feature only affects the sorting order.
\end{quote}

This command ensures that words which differ only in (say) diacritical
marks are adjacent in an alphabetically ordered dictionary.
Words will always be distinguished by their *DEFERRED letters, each will have
a separate entry in the vocabulary table.
Note that whenever two words differ {\em only} in deferred letters,
their sorting order is determined by the order of the deferred
letters on the *LETTERS command.

\subsection{The *SEPARATORS Command}
This command is optional and when present informs the package of
those characters which separate one word from the next.  When this command
is absent every character that is not declared by the *LETTERS command
is automatically assumed to be a separator.  The symbols one would use
to separate one word from the next might be the fullstop, comma,
semicolon, etc.  The CLOC package always takes a `space' to be a
separator.

\subsubsection{Example}
\begin{quote}
*SEPARATORS~~~~~? ! ; .
\end{quote}

\subsubsection{General form}
\begin{quote}
*SEPARATORS~~~~~{\em separator characters}
\end{quote}

The order in which {\em separator characters} appear on this command
is of no significance.  Note that a character must (and cannot) be declared
both as a {\em letter character} and as a {\em separator character} at one and
the same time.  Those characters which are neither {\em letter characters} nor
{\em separator characters} will be assumed to signify `spaces' and will be
interpreted as if they were declared on the following control statement.

\subsection{The *READ AS SPACE Command}
This command is optional and when present informs the package of
those characters which signify a space.  These characters although
present in the text data will be assumed to stand for the space
character and will be printed as such when concordances and collocations
are produced.

\subsubsection{Example}
\begin{quote}
*READ AS SPACE~~%
\end{quote}

\subsubsection{General form}
\begin{quote}
*READ AS SPACE~~{\em space characters}
\end{quote}

The order in which the {\em space characters} appear on this command is
of no significance.  This command can be used to remove punctuation
marks from a text or to cause one word to be read as several.  For
example, if the text contained N'EST\%PAS, the package would read it as
two words N'EST and PAS, and would print it as N'EST PAS.  If the \% sign
were declared as a (padding) {\em letter character} instead of on the
*READ~AS~SPACE command, N'EST\%PAS would be read as single {\em word}, and
printed as N'EST\%PAS.

\subsection{The *IGNORE Command}
This command is optional and when present informs the package of
those characters which are to be {\em totally ignored} when the text is read.

\subsubsection{Example}
\begin{quote}
*IGNORE~~~~~~~~@ /
\end{quote}

\subsubsection{General form}
\begin{quote}
*IGNORE~~~~~~~~{\em ignore characters}
\end{quote}

The order in which the {\em ignore characters} appear on this command
is of no significance.  This command can be used to ignore
characters which were placed in the text for special purposes.  As an
example one could cause `house-wife' to be read as if it were `housewife'
by declaring `-' as an {\em ignore character}.

\section{SAVING TEXT FILES}
\subsection{The Itemization Process}
The package treats the text as a series of {\em words} separated from each
other by {\em separators}.  Thus:

~~~~~~~~~~text:~~{\em separator word separator ... word separator}

The composition of {\em words} and {\em separators},
and the method of extracting
them from a text, are chosen by the ITEMIZE USING command and its
subsidiary commands.  Every time a CLOC job is run the text will be read
{\em word} by {\em word}, and carefully saved in a special form which permits
rapid production of concordances and collocations.  Whenever the same text is
to be examined several times it is clearly desirable to use this special
form and save computer time by not reading the same text over and over
again. They cause text to be stored in, and returned from, the
computer's filing system.  Their function is to bypass the text reading
stage and so allow computer time to be saved when many analyses are
performed on one file of text.  Further information on the filing system
can be obtained from your local computer centre.

\subsection{The SAVE TEXT command}
The SAVE TEXT command causes itemized text to be placed in a
permanent file, named {\em filename}.

\subsubsection{General form}
\begin{quote}
SAVE TEXT~~~~~~~~~~~{\em filename}
\end{quote}

\subsection{The GET TEXT command}
The GET TEXT command causes itemized text to be retrieved from
a permanent file, named {\em filename}, previously created by the SAVE TEXT
command.

\subsubsection{General form}
\begin{quote}
GET TEXT~~~~~~~~~~~~{\em filename}
\end{quote}

\subsection{Examples}
On one run of the package the following commands are sufficient to
read the text and store it in the special form.
\begin{quote}
\begin{verbatim}
ITEMIZE USING   CLOC
*LETTERS        abcdefghijklmnopqrstuvwxyz
SAVE TEXT       MYFILE
FINISH
\end{verbatim}
\end{quote}

Once the file of text has been saved, the following jobs could be run
in which the GET TEXT command replaces the itemizing instructions
in the previous example.
\begin{quote}
\begin{verbatim}
GET TEXT        MYFILE
EVERY WORD
WORDLIST        ALPHA
FINISH
\end{verbatim}
\end{quote}
and on some other run one could write:
\begin{quote}
\begin{verbatim}
GET TEXT        MYFILE
SELECT WORDS
*PATTERN        *.ing
CONCORDANCE     KWIC, CITE 4 BY 4
FINISH
\end{verbatim}
\end{quote}

\section{The OUTPUT DETAILS Command (optional)}
This command allows you to choose the maximum line width for
wordlists and citations to suit your particular lineprinter or
terminal device.
\begin{quote}
When this command is absent OUTPUT DETAILS~~WIDTH 120 is assumed.
\end{quote}

\subsection{Example}
\begin{quote}
OUTPUT DETAILS~~~~~~~~~~~~WIDTH80
\end{quote}

\subsection{General form}
\begin{quote}
OUTPUT DETAILS~~~~~~~~~~~~WIDTH{\em number}\\
Where no more than {\em number} character positions will be reserved on the
output device.  All word lists and citations will be packed into a line of this
width.
\end{quote}


\section{WORD SELECTION COMMANDS}
This section describes how one can select, from the vocabulary of
the text, a collection of words for analysis.  This collection is
used by subsequent commands when performing sorting, and
producing concordances and collocations.  One can choose the entire
vocabulary or select a portion of it.  An exclusion facility is
provided which operates on the complete vocabulary or on the portion
selected.

\subsection{Set Descriptions}
\label{sets}
These commands allow you to specify portions of the vocabulary
used in your text.
These commands all have a star symbol(*) in column 1, showing they
are subsidiary to the previous unstarred command. Several ways of
choosing words are provided. 
\begin{itemize}
\item By frequency of occurrence -- using the \verb/*FREQUENCY/ command
\item By an explicit list -- using the \verb/*LIST OF WORDS/ command
\item By a pattern -- using the \verb/*PATTERN/ command
\end{itemize}

\subsection{The *FREQUENCY command}
\label{freq}
This command is used to choose a set of words each member of which
has a particular frequency of occurrence or lies in a given frequency
range.

\subsubsection{Examples}
\begin{itemize}
\item  \verb/*FREQUENCY      (100 TO 500)/\\
      This command will select only those words which occur between 100
      and 500 times inclusive.
\item  \verb/*FREQUENCY      1 OR 4 OR >50/\\
      This command will select words which occur exactly once, exactly
      four times, or more than 50 times.
\end{itemize}

\subsubsection{General form}
\begin{quote}
*FREQUENCY         {\em expression}
\end{quote}
where {\em expression} is one or more {\em term}s connected by OR symbols.
And a {\em term} is one of the following:

\begin{itemize}
\item  {\em integer}                 for example 10\\
    only words occurring exactly {\em integer} times will be selected.
\item \verb/>/{\em integer}                 for example \verb/>/10\\
    only words occurring more than {\em integer} times will be selected.
\item \verb/</{\em integer}                 for example \verb/</10\\
    only words occurring less than {\em integer} times will be selected.
\item ({\em integer1} TO {\em integer2})  for example (100 TO 500)\\
    only words lying in the range {\em integer1} to {\em integer2} inclusive
    will be selected. Note that {\em integer1} must be smaller than
    {\em integer2}.
\end{itemize}

\subsection{The *LIST OF WORDS command}
This command allows one to specify a set of words of interest by supplying
them explicitly.
Note that when the CLOC itemizing strategy
is in use, each item in the explicit list will be mapped to a `word'. Thus you
do not need to supply the exact case of the letters nor include padding
letters.

\subsubsection{Example}
\begin{quote}
*LIST OF WORDS      this that me you
\end{quote}

\subsubsection{General form}
\begin{quote}
*LIST OF WORDS      {\em list}
\end{quote}
where {\em list} is one or more words separated from each other by one or
more spaces.

\subsection{The *PATTERN command}
This command specifies a skeletal form of a word, and causes the package
to select only those words which match the specified pattern.  Two
reserved characters are used within a pattern:-
\begin{itemize}
\item A dummy-symbol      which is .\\
    The dummy-symbol stands for {\em any} letter.
\item A variable-symbol   which is *\\
    The variable-symbol stands for `{\em any sequence} of letters, including
none at all'.
\end{itemize}
These reserved characters can be used in combination with the {\em letter
characters} defined by the word definition commands, to construct a pattern.

\subsubsection{Examples}
\begin{itemize}
\item      *PATTERN~~~~~run*\\
All words which start with `run' are selected.
\item      *PATTERN~~~~~*ing\\
All words which end with `ing' are selected.
\item      *PATTERN~~~~~pre*ed\\
All words which start with `pre' and end in `ed' are selected.
\item      *PATTERN~~~~~*ing~~~*ed\\
All words ending with `ing' together with all ending with `ed'
This is equivalent to having two *PATTERN commands one containing
`*ing' and the other `*ed'
\item      *PATTERN~~~~~a*~~~b*~~~c*
All words starting with `a' together with all starting with `b'
together with all starting with `b' are chosen. This is equivalent to
having three *PATTERN commands.
One can use this feature to produce a full concordance in sections; first the
`a', `b', and `c's then `d', `e', `f's etc.
\item     *PATTERN~~~~~....\\
    All four letter words will be chosen.
\item     *PATTERN~~~~~.h.a...\\
    All six letter words with 2nd letter `h' and 4th letter `a' will
    be selected.
\item     *PATTERN~~~~~*...ing
    All words of {\em at least} six letters which end in `ing' will be
    picked out.
\end{itemize}

\subsubsection{The DUMMYaVARIABLEb option}
This option must be used whenever a given pattern is to contain `*' or
`.' as letters.  The revised symbols apply for the current *PATTERN command
only.

\begin{itemize}
\item   *PATTERN~~~~~DUMMY?VARIABLE-~~~~*run-\\
       The `?' temporarily replaces `.' as the dummy-symbol; the `-'
       temporarily replaces `*' as the variable-symbol.  All words which start
       with `*run' are selected.
\item   *PATTERN~~~~~DUMMY.VARIABLE?~~~~?...ing*\\
       All words {\em at least} seven letters long and ending with `ing*' are
       selected.
\end{itemize}

\subsubsection{General Forms}
\begin{itemize}
\item     *PATTERN~~~~~pattern1 pattern2 pattern3 etc.\\
      Note that DUMMY.VARIABLE* is implied before pattern1.
\item     *PATTERN~~~~~DUMMYaVARIABLEb pattern1 pattern2 pattern3 etc.\\
      The character `a' becomes the new dummy-symbol, overriding `.',
      the character `b' becomes the new variable-symbol, overriding `*'.
\end{itemize}

Notes:
\begin{itemize}
\item   At least one pattern must appear on the command.
\item   A pattern consists of {\em letters},the dummy-symbol, and the variable
symbol in any combination.
\item   When the CLOC itemizing strategy is in use each explicit
        pattern will be carefully mapped to one which will match
        the various `words' in the vocabulary. Thus the pattern `run*'
        will match `run' `running' `Run' `RUNNING' etc. Padding letters will
        be ignored since the vocabulary words do not contain them.
        For example:-
        \begin{itemize}
\item if $'$ was a padding letter then .... would match `don$'$t'
\item if $'$ was a deferred letter then .... would NOT match `don$'$t'
        \end{itemize}
This is because padding letters are removed before the
word is stored in the vocabulary table, so it looks like `dont'.
In practice it is sufficient for you to examine the vocabulary table
printed using the WORDLIST command (q.v.) to find out what `words'
are in the vocabulary.
\end{itemize}

\subsection{Combining Set Descriptions}
Set descriptions can follow each other. The vocabulary selected is
sum of the vocabularies chosen by the separate set descriptions.
For example:-
\begin{quote}
\begin{verbatim}
*FREQUENCY          (100 TO 500)
*LIST OF WORDS      me you we they
*PATTERN            ...ing
\end{verbatim}
\end{quote}
The above sequence of commands defines a set of words which
contain:- all words of frequency 100 to 500 inclusive, the words
`me' `you' `we' `they', and all six letter words which end in `ing'.

\subsection{Choosing a vocabulary}
\subsubsection{The EVERY WORD Command}
    The command EVERY WORD specifies the entire vocabulary of the text.
\subsubsection{The SELECT WORDS Command}
    The SELECT WORDS command is used, in conjunction with
    several subsidiary commands, to define a given portion of the vocabulary.
\subsubsection{The EXCLUDING Command}
    The EXCLUDING command can be used to remove
    unwanted words and reduce the size of the above collection.
\subsubsection{The INCLUDING Command}
    The  command INCLUDING is provided in case your exclusion
    commands remove too many words.

\subsubsection{Combining the above commands}
The set of words defined using the SELECT WORDS, EXCLUDING, or INCLUDING
commands is specified by way of several subsidiary commands.  These
are termed {\em set description} commands and are described in
section \ref{sets}.

To choose a collection of words one can use either of the following
two constructions, without supplying an exclusion list.

\begin{tabbing}
(a)   \=EVERY WORD~~~~~\=The entire vocabulary is selected\\
\\
(b)   \>SELECT WORDS   \>The following {\em set description}\\
      \>{\em set description} \>describes the words to be used.\\
\\
The exclusion list can be placed after either of the above constructions\\
to give the following alternatives.\\
\\
(c)   \>EVERY WORD     \>The entire vocabulary\\
      \>EXCLUDING      \>excluding\\
      \>{\em set description}\>this {\em set description} is selected.\\
\\
(d)   \>SELECT WORDS   \>The words specified in\\
      \>{\em set description1}\>{\em set description1} are used,\\
      \>EXCLUDING      \>excluding those words in\\
      \>{\em set description2}\>{\em set description2}.\\
\\
(e)   \>EVERY WORD     \>The entire vocabulary\\
      \>EXCLUDING      \>excluding\\
      \>{\em set description1}\>this {\em set description1},\\
      \>INCLUDING      \>but including\\
      \>{\em set description2}\>{\em set description2}\\
\\
(f)   \>SELECT WORDS   \>The words specified in\\
      \>{\em set description1}\>this {\em set description1}\\
      \>EXCLUDING      \>excluding\\
      \>{\em set description2}\>this {\em set description2}\\
      \>INCLUDING      \>but including\\
      \>{\em set description3}\>the {\em set description3}
\end{tabbing}

The commands EVERYWORD and SELECTWORDS {\em set description} define
a working set of words. You can then use the EXCLUDING commands to
remove words from the working set, and the INCLUDING commands to
add words to the working set. You can repeat the EXCLUDING and
INCLUDING commands as often as you need to get precisely the
collection of words that you want.

\subsubsection{Examples}
\begin{itemize}
\item SELECT WORDS\\
\relax *PATTERN~~~~~~~~~~~~*ing\\
    All words that end with `ing' are selected.
\item SELECT WORDS\\
\relax *PATTERN~~~~~~~~~~~~*ing\\
EXCLUDING\\
\relax *LIST OF WORDS~~~running jumping\\
   All words ending with `ing' apart from `running' and `jumping' are chosen.
\item EVERY WORD\\
EXCLUDING\\
\relax *PATTERN~~~~~~~~~~~~*ing\\
INCLUDING\\
\relax *LIST OF WORDS~~~~running jumping\\
   Here we choose the whole vocabulary less the words ending with `ing',
   but with `running' and `jumping' included.
\end{itemize}

\section{TASK SELECTION COMMANDS}
\subsection{sorting criteria}
\label{sorting}
Some commands such as \verb/WORDLIST/ will print your selected
vocabulary in ascending alphabetic order unless you specify otherwise.
These commands allow you to change the sorting criteria by means of
a keyword described below:-
\begin{description}
\item[ALPHA]
This causes the package to sort the words into ascending alphabetic
order.  The collating order for letters is taken from the word
definition commands.

\item[DALPHA]
This causes the package to sort the words into descending alphabetic
order.  The collating order for letters is taken from the word
definition commands.

\item[REVALPHA]
This causes the package to sort the selected words into reverse
alphabetic order, in which words with similar endings sort together. The
collating order for letters is taken from the word definition commands.

\item[AFREQ]
This causes the package to sort the words into ascending frequency
order. Words having the same frequency of occurrence will be sorted
in ascending alphabetic order.

\item[DFREQ]
This causes the package to sort the selected words into descending
frequency order.  Words having the same frequency of occurrence will be
sorted in ascending alphabetic order.

\item[FIRST]
This causes the package to sort the selected words in the order in
which they {\em first occur} in the text.  Note that words are printed
across the page.

\item[LAST]
This causes the package to sort the selected words in the order in
which they {\em last occur} in the text.  Note that words are printed
across the page.

\item[ALENGTH]
This causes the package to sort the selected words in ascending
length order, which is in order of their length in characters
ignoring any deferred (or padding) letters. Words of equal length
will be sorted in ascending alphabetic order.

\item[DLENGTH]
This causes the package to sort the selected words in descending
length order, which is the descending order of their length in
characters ignoring any deferred (or padding) letters. Words of equal
length will be sorted in ascending alphabetic order.

\item[AXLENGTH]
This causes the package to sort the selected words in ascending
extended length order, which is the order of their length in
characters including any deferred (or padding) letters. Words of equal
length will be sorted in ascending alphabetic order.

\item[DXLENGTH]
This causes the package to sort the selected words in descending
extended length order, which is the descending order of their length in
characters including any deferred (or padding) letters. Words of equal
length will be sorted in ascending alphabetic order.
\end{description}

\subsection{style}
\label{style}
\begin{quote}
{\em style} is the type of citations that are required.
\end{quote}

\begin{itemize}
\item   KWIC --- key word on context, in which the word of interest is
      centralised on the print line.  CENT can be used as a synonym
for KWIC.
\item   LEFT --- in which the line of context is printed as far to the left
      as possible.
\item   NULL --- no citations of any kind will be printed, but the frequency
      counts will be printed.
\end{itemize}

\subsection{citation width}
\label{width}
\begin{quote}
{\em citation width} indicates the amount of context to be printed.  This
takes the form:
\end{quote}

\begin{itemize}
\item  CITE {\em integer1} BY {\em integer2}\\
    in which {\em integer1} words are printed before the keyword, and
    {\em integer2} words are printed after the keyword.
\item  CITE FROM{\em char1}TO{\em char2}INCLUSIVE\\
    This option causes the package to print the citations between two
    given characters {\em char1} and {\em char2}.  The left context begins with
    character {\em char1}, and the right context ends with {\em char2}.
    When INCLUSIVE is present the characters {\em char1} and {\em char2} are
    removed from the printed line.
\item  CITE FROM{\em char1}TO{\em char2}EXCLUSIVE\\
    This option causes the package to print the citations between two
    given characters {\em char1} and {\em char2}.  The left context begins with
    character {\em char1}, and the right context ends with {\em char2}.
    When EXCLUSIVE is present the characters {\em char1} and {\em char2} are
    contained in the printed line.
\end{itemize}

\subsection{offset}
\label{offset}
\begin{quote}
{\em offset} is optional and when present takes the form:
\end{quote}

\begin{itemize}
\item ABOUT NODE+{\em integer}\\
    This option allows citations to be printed about words near to the keyword.
    For example, ABOUT NODE+1 causes all citations to be printed as if the word
    to the right of the keyword was being used for citations.
\item ABOUT NODE-{\em integer}\\
    This option allows citations to be printed about words near to the keyword.
    For example, ABOUT NODE-1 causes all citations to be printed as if the word
    to the left of the keyword was being used for citations.
\item ABOUT COLLOCATE+{\em integer}\\
    {\em This option is only relevant when using the COLLOCATIONS command.}
    It allows citations to be printed about words near to the collocate.
    For example, ABOUT COLLOCATE+1 causes all citations to be printed as if
    the word to the right of the collocate was being used for citations.
\item ABOUT COLLOCATE-{\em integer}\\
    {\em This option is only relevant when using the COLLOCATIONS command.}
    It allows citations to be printed about words near to the collocate.
    For example, ABOUT COLLOCATE-1 causes all citations to be printed as if
    the word to the left of the collocate was being used for citations.
\end{itemize}

\subsection{references}
\label{references}
\begin{quote}
{\em reference} allows each citation to be prefixed with portions of references
which are embedded in the text data.  The instruction takes the form:
\end{quote}

\subsubsection{Example}
\begin{quote}
REFS A4P2L6
\end{quote}

\subsubsection{General Form}
\begin{itemize}
\item REFS {\em letter number letter number ... letter number }\\
    The {\em letter} must be from A to Z, and identifies an embedded text
    reference.  The number of characters printed for the reference is given
    by {\em number}.  When printed, each reference will be separated from the
    next by one space.
\item NOREFS\\
   When this keyword is present no text references of any kind will be printed.
\item LINEREFS0\\
   The first line of your text known as `line 0'.
\item LINEREFS1\\
   The first line of your text known as `line 1'. This is the default.
\item WORDREFS0\\
   The first word of your text known as `word 0'.
\item WORDREFS1\\
   The first word of your text known as `word 1'. This is the default.
\end{itemize}

\subsubsection{Defaults}
\begin{itemize}
\item When {\em sorting criterion} is absent, ALPHA is assumed
\item When {\em style} is absent, KWIC is assumed
\item When {\em citation width} is absent, CITE 4 BY 4 is assumed
\item When {\em offset} is absent, ABOUT NODE+0 is assumed
\item When {\em reference} is absent, an absolute record number is used
      and LINEREFS1,WORDREFS1 is assumed.
\end{itemize}

\subsection{Task Summary}
The following commands operate on the previously selected
collection of words.  Each command specifies an action to be
performed. The following tasks can be selected.
\begin{description}
\item[WORDLIST] sorting the chosen words into alphabetic and/or frequency
      order
\item[INDEX] printing a word-index
\item[CONCORDANCE] producing a concordance of the chosen words
\item[CO-OCCURRENCE] + finding co-occurrences of words
\item[COLLOCATIONS] discovering the collocations within the context of the
      chosen words.
\item[STATISTICS] count aspects of the text
\item[WRITETEXT] + write out the itemized text
\item[NEWLINE] + output a newline
\item[NEWPAGE] + output a newpage
\item[MESSAGE] + output a message
\item[NOTE] + include a comment
\item[FINISH] + terminate the run of the package
\end{description}
Each command can be included as often as required. Those marked + above
do not need to be preceded by any word selection statements.

\subsection{The WORDLIST Command}
This command causes the package to sort the previously chosen
collection of words into order, and print them.
The type of sorted list that is produced is determined by the keyword
in the specification field.  This allows one to produce word counts
in alphabetic order, reverse alphabetic order, etc.  These are printed
across the page, the number per line is determined by the maximum
word length and output line width.  In all word lists each word is
preceded by its frequency of occurrence.

\subsubsection{Examples}
\begin{itemize}
\item   \verb/WORDLIST      ALPHA/\\
    An alphabetic wordlist, is produced.
\item   \verb/WORDLIST      REVALPHA/\\
    A reverse alphabetic wordlist, i.e. one in rhyming order, is
    produced.
\item   \verb/WORDLIST      AFREQ/\\
    A wordlist in ascending frequency order is printed.
\end{itemize}

\subsubsection{General form}
\begin{quote}
WORDLIST       {\em sorting criterion}
\end{quote}

where the {\em sorting criterion} is one of the options described
in Section \ref{sorting}.

In each of the above cases, each word printed is preceded by its
frequency of occurrence in the text. By default, the words and their
frequencies are printed across the page, rather than in columns. This
is done so that you can easily write
a program (say in SNOBOL or BASIC) to reformat the output from
the CLOC package as suitable data for another package, say for
statistical analysis or graph plotting.

\subsection{The INDEX Command}
This command instructs the package to print a word index
of the selected words.  The parameters in the specification
field allow you to presort the keywords, and optionally supply the form
of text reference that will be used.

\subsubsection{Examples}
\begin{itemize}
\item   \verb/INDEX/\\
    Produces a word index in ascending alphabetic order.
    Each reference to a line is specified by a simple line number.
\item   \verb/INDEX            ALPHA/\\
    Produces a word index in ascending alphabetic order.
    Each reference to a line is specified by a simple line number.
\item   \verb/INDEX            REVALPHA,REFS P2 L2/\\
    A reverse alphabetic word index is produced.
    Each reference gives information about `Page number' and `Line
    within page' assuming that \verb/P/ and \verb/L/ references are included in
    the text for each page.
\item   \verb/INDEX            NOREFS/\\
    An alphabetical word index is produced.
    No text references of any kind are printed.
\end{itemize}

\subsubsection{General Form}
\begin{quote}
INDEX           {\em sorting criterion , reference}
\end{quote}

where the {\em sorting criterion} is one of the options described
in Section \ref{sorting}.
The chosen set of keywords are sorted into this order before the
word index is produced.

{\em reference} allows each word to be referenced with portions of references
which are embedded in the text data.  See Section \ref{references} for details.

\subsection{The CONCORDANCE Command}
This command instructs the package to print a concordance
of the selected words.  The parameters in the specification
field allow the user to presort the keywords; to
choose the citation style; to select a citation width; and optionally
supply a text reference.

\subsubsection{Examples}
\begin{itemize}
\item   \verb/CONCORDANCE/\\
    The keywords are alphabetically sorted.
    Keyword in context citations are printed which
    have four words on either side of the word of interest.  Each line is
    identified by a simple record number.
\item   \verb/CONCORDANCE  ALPHA,KWIC,CITE 4 BY 4/\\
    The keywords are alphabetically sorted.
    Keyword in context citations are printed which
    have four words on either side of the word of interest.  Each line is
    identified by a simple record number.
\item   \verb/CONCORDANCE  REVALPHA,CITE 6 BY 6,REFS P2 L2/\\
    A reverse alphabetic KWIC concordance is produced with six words
    of context on either side of the keyword.  Each line printed is
    prefixed by text reference information giving `Page number' and `Line
    within page' assuming that \verb/P/ and \verb/L/ references are included
    in the text for each page.
\item   \verb/CONCORDANCE  CITE FROM.TO.INCLUSIVE/\\
    An alphabetical KWIC concordance is printed.  The keyword is
    surrounded by as many words as possible up to and including a `.'
    character.  By this means a sentence of context is printed.
\item  \verb+CONCORDANCE  REVALPHA,LEFT,CITE FROM/TO/EXCLUSIVE,REFS S2 P3 L2+\\
    Assuming that `/' has been declared as a `newline' character
    (see INPUT DETAILS command), this example will print a reverse
    alphabetic concordance.  Each citation will consist of a full line
    of context, left justified and prefixed with `section', `page' and
    `line number' information, assuming that \verb/S/, \verb/P/ and \verb/L/
    references have been included in the text.
\item   \verb/CONCORDANCE  REVALPHA,CITE 6 BY 6,NOREFS/\\
    A reverse alphabetic KWIC concordance is produced with six words
    of context on either side of the keyword.  No text references of
    any kind will precede the citations.
\item   \verb/CONCORDANCE  CITE 4 BY 4, ABOUT NODE-1/\\
    The keywords are alphabetically sorted.
    Keyword in context citations are printed which
    have four words on either side of the word of interest.  Each line is
    identified by a simple record number. The citations will be printed
    with the word before the keyword centralised on the line.
\end{itemize}

\subsubsection{General Form}
\begin{quote}
CONCORDANCE   {\em sorting criterion, style, citationwidth, offset, reference}
\end{quote}

where the {\em sorting criterion} is one of the options described
in Section \ref{sorting}.
The chosen set of keywords are sorted into this order before the
concordance is produced.

{\em style} is the type of concordance required.
See Section \ref{style} for details.

{\em citationwidth} indicates the amount of context to be printed. This is
described in Section \ref{width}.

{\em offset} allows you to choose which node will be centred. See Section
\ref{offset} for details.

{\em reference} allows each citation to be prefixed with portions of references
which are embedded in the text data. This is described in Section \ref{references} .

\subsection{The CO-OCCURRENCE Command}
This command is used when you know two or more words and need
to study how they occur in a text.  The parameters in the specification
field allow you to choose a style for presenting the results; to
select a citation width and to optionally supply a text reference.
Subsidiary commands enable you to choose word pairs
or phrases of interest and also to choose a series of words
separated by an arbitrary word distance.  {\em Note}:  The CO-OCCURRENCE
command need not be preceded by any word selection commands.

\subsubsection{Examples}
\begin{itemize}
\item
\begin{verbatim}
CO-OCCURRENCE
*PHRASE         you are
*PHRASE         now is the winter
\end{verbatim}
    This produces 4 BY 4 KWIC citations of positions in a text in
    which the phrase `you are' occurs.  After these have been found, all
    occurrences of the phrase `now is the winter' are found.  In both
    cases punctuation is ignored, hence all examples are discovered.
\item
\begin{verbatim}
CO-OCCURRENCE   CITE 8 BY 8, REFS P2 L2
*PHRASE         just man
*SERIES         a UPTO6 goat
*SERIES         how GAP2 see
*SERIES         he UPTO3 miner GAP1 and
\end{verbatim}
    This gives 8 BY 8 citations are printed centralised on the page, each
    prefixed with a 2 figure page number and 2 figure line number.  After
    printing all occurrences of the phrase `just man', the three *SERIES
    commands are obeyed in order.  The first command finds all occurrences
    of `a ... goat' where `a' and `goat' are separated from each other by
    0, 1, 2, 3, 4, 5, or 6 words of context.  The second command finds all
    occurrences of `how' and `see' separated from each other by {\em precisely}
    two arbitrary words.  The third *SERIES command shows how you can mix
    the UPTO{\em number} and GAP{\em number} options within one specification.
    This example will find all occurrences of `he' `miner' `and' where `he'
    and `miner' are separated by 0, 1, 2 or 3 arbitrary words and with
    `miner' and `and' separated by exactly 1 arbitrary word.
\item
\begin{verbatim}
CO-OCCURRENCE
*PATTERN        *ing *ly
\end{verbatim}
    This shows how you can also put a list of patterns which will be
    searched in the order they are given. This represents a skeletal form
    of a phrase.
\item
\begin{verbatim}
CO-OCCURRENCE
*CHOICE         be is as was were
\end{verbatim}
    This example shows how you can print the citations of the words BE IS
    AS WAS WERE as they are found in the text. 
\end{itemize}

\subsubsection{General Form}
\begin{itemize}
\item CO-OCCURRENCE~~~{\em style, citation width, offset, reference, counting}
\item *PHRASE~~~~~~~~~{\em list}
\item *SERIES~~~~~~~~~{\em word type number word ... type number word}
\item *PATTERN~~~~~~~~{\em pattern1 pattern2 pattern3} etc.
\item *PATTERN~~~~~~~~DUMMYaVARIABLEb {\em pattern1 pattern2 pattern3} etc.
\item *CHOICE~~~~~~~~~{\em choices}
\end{itemize}

{\em style} is the type of citation required. This is described in
Section \ref{style} .

{\em citation width} specifies the amount of context to be printed.
This is described in Section \ref{width}

{\em offset} allows you to choose which node will be centred. See Section
\ref{offset} for details.
When citations are printed the node for offset purposes is the first word
of a *PHRASE or a *SERIES or a *PATTERN. This option allows you to
shift the citation left or right as required.

{\em reference} allows each citation to be prefixed with portions of references
which are embedded in the text data. This is described in Section \ref{references} .

{\em counting} allows you to produce frequency counts of the citations selected
    by the *PHRASE *SERIES *PATTERN or *CHOICE commands. You choose from:-
\begin{itemize}
\item COUNT --- a citation frequency count precedes the citations.
\item NOCOUNT --- no frequency count is printed. This option is the default.
\end{itemize}
\begin{quote}
{\em NOTE:} You can use CO-OCCURRENCE  NULL,COUNT if you just want the
  frequency counts and not the actual citations.
\end{quote}

{\em type} {\em number} is either UPTO{\em number} or GAP{\em number}, and where

\begin{quote}
UPTO{\em number} means 0, 1, 2, 3.. {\em number} of arbitrary words may
occur at this position.

GAP{\em number} means exactly {\em number} arbitrary words occur at
this position.
\end{quote}

{\em list} is one or more words separated from each other by spaces.
These will be treated as consecutive words.

{\em choices} is one or more words separated from each other by spaces.
These will be treated as alternative possibilities at a single position
in the text rather than as a phrase.

\begin{quote}
{\em Notes}
\end{quote}
\begin{itemize}
\item   Each word must be separated from the {\em type} and {\em number} by at
least one space.

\item   The {\em number} can be 0, in which case UPTO0 and GAP0 are equivalent
and indicate that the two words are to be adjacent.  *PHRASE
is the degenerative case of a *SERIES in which all {\em numbers} are
zero.

\item   *PHRASE and *SERIES and *PATTERN and *CHOICE commands can be
repeated as often as required.
\end{itemize}

\subsection{The COLLOCATIONS Command}
This command instructs the package to examine the collocates in
the context surrounding each selected word.  Those collocates which
are found to have a significant affinity to the selected word will have
their context printed.  This option allows closely associated pairs
of words to have their context printed.  The occurrence of a collocate
will be counted whenever it occurs in a range several words to the left
or to the right of the selected word.  This region is termed a {\em span},
the size of which can be chosen using a subsidiary command.

\subsubsection{Examples}
\begin{itemize}
\item   \verb/COLLOCATIONS/\\
    The keywords are alphabetically sorted.
    The collocates are printed as keyword in context citations which
    have four words on either side of the word of interest.  Each line is
    identified by a simple record number.
\item   \verb/COLLOCATIONS  ALPHA,KWIC,CITE 4 BY 4/\\
    The keywords are alphabetically sorted.
    The collocates are printed as keyword in context citations which
    have four words on either side of the word of interest.  Each line is
    identified by a simple record number.
\item   \verb/COLLOCATIONS  REVALPHA,CITE 6 BY 6,REFS P2 L2/\\
    Reverse alphabetic KWIC citations are produced with six words
    of context on either side of the keyword.  Each line printed is
    prefixed by text reference information giving `Page number' and `Line
    within page' assuming that \verb/P/ and \verb/L/ references are included
    in the text for each page.
\item   \verb/COLLOCATIONS     CITE FROM.TO.EXCLUSIVE, ABOUT COLLOCATE-1/\\
    Alphabetical KWIC citations are printed with the word before the
    collocate as the central keyword.  The keyword is
    surrounded by as many words as possible up to and including a `.'
    character.  By this means a sentence of context is printed.
\item  \verb/COLLOCATIONS     FIRST,CITE 6 BY 6, ABOUT COLLOCATE/\\
    The nodes are sorted in order of their first occurrence in the text.
    Each citation will be centred on the collocate and have 6 words of context
    on either side.
\item   \verb/COLLOCATIONS     FIRST,CONDENSED/\\
    The nodes are sorted in order of their first occurrence in the text.
    The collocated will be printed in a simple table which will give the
    frequency and word of the node, it's collocate and the pair.
\end{itemize}

\subsubsection{General forms}
\begin{itemize}
\item 
COLLOCATIONS {\em sorting criterion, style, citation width, offset, reference}
\item COLLOCATIONS {\em sorting criterion}, CONDENSED
\item COLLOCATIONS {\em sorting criterion}, FULLCONDENSED
\end{itemize}

where {\em sorting criterion} is described in Section \ref{sorting}
The chosen set of keywords are sorted into this order before the
collocations are produced.

{\em style} is the type of citation required.  This is described in Section
\ref{style}.

{\em citation width} indicates the amount of context to be printed.  This
is described in Section \ref{width}.

{\em offset} allows you to choose which node or collocate will be centred.
See Section \ref{offset} for details.

{\em reference} allows each citation to be prefixed with portions of references
which are embedded in the text data.  This is described in Section
\ref{references}.

The CONDENSED option causes the discovered collocates to be listed in
a simple tabular form.  This gives on quick look at an author's word
associations, allowing one to choose accurately which nodes to select
and which collocates to reject.  The following example illustrates
the format of the table produced by this command.
\begin{verbatim}
condensed collocation analysis of 71 nodes
==========================================
    node               collocate           pair
    ----               ---------           ----
    6 and                6 of                4
    6 and                9 the               4
    6 and                9 a                 4
    4 in                 9 a                 4
    6 of                 9 the               4
    9 the                6 university        5
    9 the                6 of                4
    9 the                6 and               4
    6 university         9 the               5
    6 university         9 a                 4
\end{verbatim}

  The FULLCONDENSED option produces a table which also contains
the totalspan and the meanspan as follows:-
\begin{verbatim}
full condensed collocation analysis of 71 nodes
===============================================
* text has 113 running words, 71 distinct words
* span 4 by 4 restricted
* meanspan = totalspan/ (nodefrequency * (leftspan + rightspan))
    node               collocate           pair  totalspan  meanspan
    ----               ---------           ----  ---------  --------
    6 and                6 of                4          47  0.979167
    6 and                9 the               4          47  0.979167
    6 and                9 a                 4          47  0.979167
    4 in                 9 a                 4          28  0.875000
    6 of                 9 the               4          31  0.645833
    9 the                6 university        5          54  0.750000
    9 the                6 of                4          54  0.750000
    9 the                6 and               4          54  0.750000
    6 university         9 the               5          39  0.812500
    6 university         9 a                 4          39  0.812500
\end{verbatim}

\begin{quote}
{Subsidiary commands to the COLLOCATIONS command}
\end{quote}
\subsection{The *SPAN command}
This command specifies the range of searching that is done when
the package performs a collocation analysis.  The specification field
of this command defines the range which will be searched in terms
of the number of words to the left and right of the word of interest.

\subsubsection{Examples}
\begin{itemize}
\item *SPAN              4 BY 4
\item *SPAN              4 BY 4 RESTRICTED
\item *SPAN              4 BY 6 UNRESTRICTED
\end{itemize}

\subsubsection{General form}
\begin{quote}
*SPAN              {\em integer1} BY {\em integer2} {\em qualifier}
\end{quote}
where {\em integer1} indicates the number of words to be searched before the
word of interest and {\em integer2} indicates the number of words after the
word of interest. Either of these integers may be zero.

{\em qualifier} is either UNRESTRICTED or RESTRICTED.
\begin{quote}
UNRESTRICTED means that all words in the left and right span will be
counted as collocates.
\end{quote}
\begin{quote}
RESTRICTED means that when a pair of nodes are
closer that leftspan+rightspan, overlapping collocates will be counted
once only, and the node will not be counted as a collocate.
\end{quote}
Note that the {\em citation width} could be narrower than the {\em span}.  This
will cause some collocates to appear to be absent from the context,
they will however be found in the text.  Normally the {\em citation width}
should be greater than the {\em span}.

\subsubsection{Defaults}
\begin{itemize}
\item When *SPAN is absent *SPAN~~~~~~~~~~4 BY 4 UNRESTRICTED is assumed
\item When {\em qualifier} is absent UNRESTRICTED is assumed.
\end{itemize}

\subsection{The *FREQUENCY command}
This command is used to select significant collocates according
to their frequency of occurrence.  Only those collocates whose {\em collocation
frequency} is within the specified limits will have their citations
printed.  Note that the frequency of occurrence of a collocate is
different from the frequency of occurrence of the same object treated
as a {\em word}.

\subsubsection{Example}
\begin{quote}
*FREQUENCY       (100 TO 500)
\end{quote}

\subsubsection{General Form}
\begin{quote}
*FREQUENCY       {\em expression}
\end{quote}

Where {\em expression} is one or more {\em term}s connected by OR symbols.  A
{\em term} is one of the following:
\begin{itemize}
\item  {\em integer}                 for example 10\\
    only collocates occurring exactly {\em integer} times will be selected.
\item \verb/>/{\em integer}                 for example \verb/>/10\\
    only collocates occurring more than {\em integer} times will be selected.
\item \verb/</{\em integer}                 for example \verb/</10\\
    only collocates occurring less than {\em integer} times will be selected.
\item ({\em integer1} TO {\em integer2})  for example (100 TO 500)\\
    only collocates lying in the range {\em integer1} to {\em integer2} inclusive
    will be selected. Note that {\em integer1} must be smaller than
    {\em integer2}.
\end{itemize}

\subsubsection{Defaults}
\begin{quote}
When the *FREQUENCY command is absent \verb/*FREQUENCY      >1/ is assumed.
\end{quote}

\subsection{The EVERY COLLOCATE Command}
This command ensures that every collocate chosen by the *SPAN and *FREQUENCY
commands will be considered for selection.

\subsubsection{Example and general form}
\begin{quote}
EVERY COLLOCATE
\end{quote}

\subsection{The SELECTCOLLOCATE Command}
This must be followed be a series of {\em set description}s and ensures
only those words in the {\em set description}s will be considered
as collocates.

\subsubsection{Example}
\begin{quote}
\begin{verbatim}
SELECTCOLLOCATE
*LIST OF WORDS   father mother
*PATTERN         *ing
\end{verbatim}
\end{quote}
This example only selects the collocates `father' and `mother' and all
collocates which end with `ing'.

\subsection{The REJECTING Command}
This must be followed be a series of {\em set description}s and ensures
only those words in the {\em set description}s will be removed from
the collocates that have been previously chosen.
This command allows you to specify an exclusion list for
collocates.  Its function is to remove insignificant collocates.

\subsubsection{Example}
\begin{quote}
\begin{verbatim}
REJECTING
*PATTERN         run*
\end{verbatim}
\end{quote}

\subsection{The ACCEPTING Command}
This must be followed be a series of {\em set description}s and ensures
those words in the {\em set description}s will be added to
the collocates that have been previously chosen.
It is most often used to supply words
that were excluded because the REJECTING command was too restrictive.

The above commands can only be supplied in a fixed order. The order is
similar to that for {\em word selection} described earlier but this time we
use the collocate selection commands. The commands EVERY COLLOCATE or
SELECTCOLLOCATE are alternatives, only one can be chosen, but if both
are absent the command EVERY COLLOCATE is assumed. Thus we can say:-
\begin{itemize}
\item  EVERY COLLOCATE
\item  SELECTCOLLOCATE\\
       {\em set description}
\end{itemize}
The commands REJECTING or ACCEPTING can be chosen as often as necessary
so that you can get just those collocates that you want.

\subsubsection{Examples}
\begin{itemize}
\item
\begin{verbatim}
REJECTING
*LIST OF WORDS         the but and
\end{verbatim}
The specific collocates `the' `but' `and' will not be printed.

\item
\begin{verbatim}
REJECTING
*PATTERN               *ing *ed
\end{verbatim}
All collocates which end in `ing' or `ed' will not be printed.

\item
\begin{verbatim}
REJECTING
*FREQUENCY             (100 TO 700)
\end{verbatim}
All collocates whose vocabulary frequency of occurrence is
between 100 and 700 inclusive will not be printed.

\item
\begin{verbatim}
SELECTCOLLOCATE
*PATTERN               run*
\end{verbatim}
Only those collocates which start with `run' will be chosen.

\item
\begin{verbatim}
SELECTCOLLOCATE
*PATTERN               run*
REJECTING
*LIST OF WORDS         running
\end{verbatim}
All collocates which start with `run' but excluding the word `running'
will be chosen.
\end{itemize}

\subsection{The STATISTICS Command}
The subcommands of this command will count aspects of the text and produce
tables of data.
\subsection{The *PROFILE Command}
This command will produce a table containing the number of words which fall
into frequency bands and a cumulative total of those words.
\subsubsection{Examples}
\begin{itemize}
\item *PROFILE
\item *PROFILE       BINSIZE 50
\end{itemize}

\subsubsection{General form}
\begin{quote}
*PROFILE        BINSIZE {\em integer}
\end{quote}
The command will cause frequencies to be grouped into {\em integer} ranges.

\subsubsection{Default}
\begin{quote}
When {\em integer} is absent 100 is assumed.
\end{quote}

\subsection{The WRITETEXT Command}
This command will cause CLOC to print out the itemized text. Each line
can be prefixed with a text reference to the first word in each line.

\subsubsection{Examples}
\begin{itemize}
\item WRITETEXT\\
Each line of the text will contain a reference which will be a
simple record number.

\item WRITETEXT         REFS P3 L4\\
Each line of the text will contain a reference which will be a 3 character
page number P and a 4 figure line number L .
\item WRITETEXT         NOREFS\\
No references of any kind will precede each line of text.
\end{itemize}

\subsubsection{General form}
\begin{quote}
WRITETEXT      {\em reference}
\end{quote}
where {\em reference} is optional, and is described in Section \ref{references}.

\subsection{The NEWLINE Command}
This command will insert one or more newlines in the CLOC results file.
You can use this feature to widen the gap between the results from
successive tasks.

\subsubsection{Examples}
\begin{itemize}
\item NEWLINE
\item NEWLINE        1
\item NEWLINE        5
\end{itemize}

\subsubsection{General form}
\begin{quote}
NEWLINE        {\em integer}
\end{quote}
The command will cause {\em integer} newlines to be sent to the CLOC
results file.

\subsubsection{Default}
\begin{quote}
When {\em integer} is absent, a value of 1 is assumed.
\end{quote}

\subsection{The NEWPAGE Command}
This command will cause a newpage to be thrown on the CLOC results file.
\subsubsection{Example and general form}
\begin{quote}
NEWPAGE
\end{quote}

\subsection{The MESSAGE Command}
This command will send the contents of the specification field, and any
continuation, to the CLOC results file.
\subsubsection{Example}
\begin{quote}
MESSAGE        Henry the Fifth (part 1)
\end{quote}

\subsubsection{General form}
\begin{quote}
MESSAGE        {\em character sequence}
\end{quote}
The {\em character sequence} will be sent to the CLOC results file.

\subsection{The NOTE Command}
This command can be used at your convenience to insert a
commentary about the following or preceding task.  All characters
in the specification field of this command are ignored.

\subsubsection{Example}
\begin{quote}
NOTE            THIS TEXT IS TAKEN FROM WORDSWORTH
\end{quote}

\subsubsection{General Form}
\begin{quote}
NOTE            {\em character sequence}
\end{quote}
Where {\em character sequence} is totally ignored.  This information will
be printed on the CLOC diagnostic and information channel along
with the other control statements.

\subsection{The FINISH Command}
This command must be the final one in the sequence.  It informs the
package that no further commands are to follow.
\subsubsection{Example and general form}
\begin{quote}
FINISH
\end{quote}

\newpage
\appendix
\section{EXAMPLES}
 Before preparing a large volume of text, or before trying out the
 CLOC package on some prepared text, you should run the example
 programs given below.  To do this you will need to read the
 documentation on using the package on your local computer.  This
 will provide the basic information you need to know to run any CLOC
 job.  You should compare the results produced by the computer with
 those given in the example programs to check your understanding of
 the command language.  You are also recommended to vary the given
 commands in order to gain some feel for their effect.  Often
 the examples will contain all the commands you need to solve
 your given problem, in which case all you need do is supply
 your own text.
 All the examples in this section use the following extract from
 `CAUTION: LOW FLYING DUCKS' by the author.

\begin{verbatim}
          The University ; "A society of individuals living and working
     together for the advancement of learning and the dissemination of
     knowledge".  (University of York Development Plan).
          In 1617 James I received a petition requesting a University
     for York.  This was followed by a petition to Parliament in 1652,
     and a deputation to the University Grants Committee in 1947.  The
     University officially opened in 1963 with a student population
     comprising 216 undergraduates and 12 postgraduates.
          The site consisted of 190 acres of marshy land and a large
     decrepit Elizabethan mansion, Heslington Hall, destined to become
     the administration building.  Draining the saturated ground was
     accomplished by widening a natural stream and creating a fourteen
     acre artificial lake around which the University was constructed.
\end{verbatim}

 If the above were coded on punched cards using the recommendations in
 section 2 of this guide, it would look like this, the following:-

\begin{verbatim}
          =THE =UNIVERSITY : "=A SOCIETY OF INDIVIDUALS LIVING AND WORKING
     TOGETHER FOR THE ADVANCEMENT OF LEARNING AND THE DISSEMINATION OF
     =KNOWLEDGE".   (=UNIVERSITY OF =YORK =DEVELOPMENT =PLAN)  
          =IN 1617 =JAMES =I RECEIVED A PETITION REQUESTING A =UNIVERSITY
     FOR =YORK,  =THIS WAS FOLLOWED BY A PETITION TO =PARLIAMENT IN 1652,
     AND A DEPUTATION TO THE =UNIVERSITY =GRANTS =COMMITTEE IN 1947.  =THE
     =UNIVERSITY OFFICIALLY OPENED IN 1963 WITH A STUDENT POPULATION
     COMPRISING 216 UNDERGRADUATES AND 12 POSTGRADUATES.
          =THE SITE CONSISTED OF 190 ACRES OF MARSHY LAND AND A LARGE
     DECREPIT =ELIZABETHAN MANSION, =HESLINGTON =HALL, DESTINED TO BECOME
     THE ADMINISTRATION BUILDING.  =DRAINING THE SATURATED GROUND WAS
     ACCOMPLISHED BY WIDENING A NATURAL STREAM AND CREATING A FOURTEEN
     ACRE ARTIFICIAL LAKE AROUND WHICH THE =UNIVERSITY WAS CONSTRUCTED.
\end{verbatim}

 We will assume that you are using a computer that has both upper
 and lower case and that you stored the text in the form that it
 was first written. The following examples show how CLOC commands
 are put together to perform the following tasks:-
\begin{itemize}
\item Alphabetic sorting
\item The pattern feature
\item The exclusion list
\item Producing a concordance
\item Finding collocations
\end{itemize}
\newpage
           {\em Example number} 1             {\em Alphabetic Sorting}\\
           The following commands cause CLOC to read the above text and
           sort the vocabulary into ascending alphabetic order.
\begin{verbatim}
                 ITEMIZEUSING)cloc
                 *LETTERS)abcdefghijklmnopqrstuvwxyz
                 OUTPUTDETAILS)width80
                 EVERYWORD
                 WORDLIST)alpha
                 FINISH
\end{verbatim}

           The output produced by the computer is a listing of the commands,
           including the defaults and comments, and the results of the
           sorting process.

                    a)  Control statement listing
\begin{verbatim}
default                input details  width80
                       ITEMIZEUSING)cloc
                       *LETTERS)abcdefghijklmnopqrstuvwxyz
default                *separators    !"#$%&'()*+,-./0123456789:;<=>?@[\]^_`{|}~
the text contains :
     113  running words
      71  distinct words
and the maximum word length is 14 characters

                       OUTPUTDETAILS)width80
                       EVERYWORD
                       WORDLIST)alpha
                       FINISH
\end{verbatim}
                     b)   The Results.

\begin{verbatim}
table of 71 words in ascending alphabetic order
===============================================
    9 a                 1 accomplished      1 acre              1 acres         
    1 administration    1 advancement       6 and               1 around        
    1 artificial        1 become            1 building          2 by            
    1 committee         1 comprising        1 consisted         1 constructed   
    1 creating          1 decrepit          1 deputation        1 destined      
    1 development       1 dissemination     1 draining          1 elizabethan
    1 followed          2 for               1 fourteen          1 grants
    1 ground            1 hall              1 heslington        1 i     
    4 in                1 individuals       1 james             1 knowledge
    1 lake              1 land              1 large             1 learning
    1 living            1 mansion           1 marshy            1 natural    
    6 of                1 officially        1 opened            1 parliament 
    2 petition          1 plan              1 population        1 postgraduates
    1 received          1 requesting        1 saturated         1 site
    1 society           1 stream            1 student           9 the
    1 this              3 to                1 together          1 undergraduates
    6 university        3 was               1 which             1 widening
    1 with              1 working           2 york          
\end{verbatim}

\newpage
 {\em Example number} 2                 {\em The PATTERN Feature}\\
 The example illustrates how the pattern feature can be used to select, from
 the above text, words which end in a standard way. The selected words are
 then listed in alphabetic order.
\begin{verbatim}
                    ITEMIZEUSING)cloc
                    *LETTERS)abcdefghijklmnopqrstuvwxyz
                    OUTPUTDETAILS)width80
                    SELECTWORDS
                    *PATTERN) *ing
                    WORDLIST)alpha
                    FINISH
\end{verbatim}

         The output produced is a listing of the commands and the results.

                     a)  Control statement listing
\begin{verbatim}
default                input details  width80
                       ITEMIZEUSING)cloc
                       *LETTERS)abcdefghijklmnopqrstuvwxyz
default                *separators    !"#$%&'()*+,-./0123456789:;<=>?@[\]^_`{|}~
the text contains :
     113  running words
      71  distinct words
and the maximum word length is 14 characters

                       OUTPUTDETAILS)width80
                       SELECTWORDS
                       *PATTERN) *ing
                       WORDLIST)alpha
                       FINISH

\end{verbatim}
                      b)   The Results.
\begin{verbatim}
table of 9 words in ascending alphabetic order
==============================================
     1 building          1 comprising        1 creating          1 draining
     1 learning          1 living            1 requesting        1 widening
     1 working
\end{verbatim}

\newpage
 {\em Example Number} 3                      {\em The Exclusion List}

 This example shows how one can exclude a set of words from a previously
 selected set.  The resultant collection is listed in ascending alphabetic
 order.
\begin{verbatim}
                    ITEMIZEUSING)cloc
                    *LETTERS)abcdefghijklmnopqrstuvwxyz
                    OUTPUTDETAILS)width80
                    SELECTWORDS
                    *FREQUENCY) >1
                    EXCLUDING
                    *LISTOFWORDS) a the of and
                    WORDLIST)alpha
                    FINISH
\end{verbatim}
    The output produced is a listing of the commands and the alphabetically
    ordered list.

                     a)   Control statement listing
\begin{verbatim}
default                input details  width80
                       ITEMIZEUSING)cloc
                       *LETTERS)abcdefghijklmnopqrstuvwxyz
default                *separators    !"#$%&'()*+,-./0123456789:;<=>?@[\]^_`{|}~
the text contains :
     113  running words
      71  distinct words
and the maximum word length is 14 characters

                       OUTPUTDETAILS)width80
                       SELECTWORDS
                       *FREQUENCY) >1
                       EXCLUDING
                       *LISTOFWORDS) a the of and
                       WORDLIST)alpha
                       FINISH

\end{verbatim}
                    b)  The Results
\begin{verbatim}
table of 8 words in ascending alphabetic order
==============================================
    2 by                2 for               4 in                2 petition
    3 to                6 university        3 was               2 york
\end{verbatim}

\newpage
 {\em Example number} 4                    {\em Producing a CONCORDANCE}

The following commands will produce a concordance of the words selected.
The output is centralized on the page and sorted in ascending alphabetic order.
\begin{verbatim}
                   ITEMIZEUSING)cloc
                   *LETTERS)abcdefghijklmnopqrstuvwxyz
                   OUTPUTDETAILS)width80
                   SELECTWORDS
                   *PATTERN) *ed
                   CONCORDANCE) KWIC,CITE 5 BY 5
                   FINISH
\end{verbatim}
                      a)   Control statement listing
\begin{verbatim}
default                input details  width80
                       ITEMIZEUSING)cloc
                       *LETTERS)abcdefghijklmnopqrstuvwxyz
default                *separators    !"#$%&'()*+,-./0123456789:;<=>?@[\]^_`{|}~
the text contains :
     113  running words
      71  distinct words
and the maximum word length is 14 characters

                       OUTPUTDETAILS)width80
                       SELECTWORDS
                       *PATTERN) *ed
                       CONCORDANCE) KWIC,CITE 5 BY 5
                       FINISH
\end{verbatim}

                     b)   The results
\begin{verbatim}

concordance of 8 nodes
======================
node  accomplished  occurs 1 times
12       Draining the saturated ground was accomplished by widening a natural 

node  consisted  occurs 1 times
9          12 postgraduates.      The site consisted of 190 acres of marshy 

node  constructed  occurs 1 times
13         around which the University was constructed. 

node  destined  occurs 1 times
10               mansion, Heslington Hall, destined to become the 

node  followed  occurs 1 times
5           University for York.  This was followed by a petition to Parliament

node  opened  occurs 1 times
7         1947.  The University officially opened in 1963 with a student 

node  received  occurs 1 times
4              Plan).      In 1617 James I received a petition requesting a 

node  saturated  occurs 1 times
11                 building.  Draining the saturated ground was accomplished by
\end{verbatim}

\newpage
 {\em Example number} 5               {\em Finding COLLOCATIONS}

 The following commands cause the package to scan the context of
 the selected words, and to print the examples of their collocations.
 The output is centralised on the page and sorted in ascending alphabetic
 order.
\begin{verbatim}
                     ITEMIZEUSING)cloc
                     *LETTERS)abcdefghijklmnopqrstuvwxyz
                     OUTPUTDETAILS)width80
                     SELECTWORDS
                     *LISTOFWORDS) university
                     COLLOCATIONS) KWIC, CITE 5 BY 5
                     *FREQUENCY)>2
                     FINISH
\end{verbatim}
                     a)   Control statement Listing
\begin{verbatim}
default                input details  width80
                       ITEMIZEUSING)cloc
                       *LETTERS)abcdefghijklmnopqrstuvwxyz
default                *separators    !"#$%&'()*+,-./0123456789:;<=>?@[\]^_`{|}~
the text contains :
     113  running words
      71  distinct words
and the maximum word length is 14 characters

                       OUTPUTDETAILS)width80
                       SELECTWORDS
                       *LISTOFWORDS) university
                       COLLOCATIONS) KWIC, CITE 5 BY 5
default                *span          4 by 4
                       *FREQUENCY)>2
                       FINISH
\end{verbatim}
                     b) The Results
\begin{verbatim}
collocation analysis of 1 nodes (cited about node)
==================================================
node  university  occurs 6 times
collocate  the  occurs 9 times
node-collocate pair occurs 6 times
1                                      The University ; "A society of 
3       the dissemination of knowledge".  (University of York Development Plan
6                  and a deputation to the University Grants Committee in
6                  and a deputation to the University Grants Committee in
7           Grants Committee in 1947.  The University officially opened in
13        artificial lake around which the University was constructed. 

node  university  occurs 6 times
collocate  a  occurs 9 times
node-collocate pair occurs 4 times
1                                      The University ; "A society of 
4         received a petition requesting a University for York.  This was 
4         received a petition requesting a University for York.  This was 
6                  and a deputation to the University Grants Committee in

node  university  occurs 6 times
collocate  of  occurs 6 times
node-collocate pair occurs 3 times
1                                      The University ; "A society of 
3       the dissemination of knowledge".  (University of York Development Plan
3       the dissemination of knowledge".  (University of York Development Plan

node  university  occurs 6 times
collocate  in  occurs 4 times
node-collocate pair occurs 3 times
6                  and a deputation to the University Grants Committee in
7           Grants Committee in 1947.  The University officially opened in
7           Grants Committee in 1947.  The University officially opened in
\end{verbatim}

\newpage
\section{Messages Produced by the CLOC package}
Three categories of message are printed by the package, these are {\em errors},
{\em warnings}, and {\em comments}.

\subsection{Error Messages}
 These cause the run of the package to be abandoned.  Where the error is
 caused by a mistake in a command the figure ~1~ is printed under the
 faulty position on the command, ~2~ for the second error on the line,
 and so on up to ~9~ . Error messages take the form:-
\begin{quote}
 ERROR - {\em text of message}
\end{quote}
where the text is one of the following
\begin{itemize}
\item MISSING MANDATORY STATEMENT\\
                 You have forgotten to include or have mis-spelt an
                 essential command.
\item CONTROL STATEMENT ENDS PREMATURELY\\
                 The continuation field of 15 spaces was expected but
                 not found.
\item INCORRECT CONTROL STATEMENT
                 A mistake has been found on the line, the symbol \ 1
                 points to it.
\item UNKNOWN SYMBOL\\
                 The item in the specification field has not been
                 recognised.
\item CHARACTER ALREADY DEFINED\\
                 The indicated character has occurred on this or an earlier
                 line.
\item NO LETTERS PROVIDED\\
                 The *LETTERS command exists, but no letters have been
                 put on it.
\item NUMBER IS TOO LARGE\\
                 The indicated number is too large for the CLOC
                 package to use.
\item UPPER VALUE DOES NOT EXCEED LOWER\\
                 The upper value in a frequency range is smaller than
                 the lower value.
\item NO WORDS FOUND\\
                 The combination of word selection commands has chosen
                 no words.
\item FILE NOT PRODUCED BY CLOC MARK {\em mark}\\
                 The file used by the GET TEXT command was not
                 produced by an earlier run of the package.
\item ABOVE STATEMENT NOT EXPECTED\\
                 This line may be a mis-spelt or spurious command.
\item SYMBOL NOT ALLOWED IN THIS CONTEXT\\
                 The indicated symbol is not permitted there.
\item CAPACITY EXCEEDED\\
                 The text to be processed contains more words than
                 the package is able to handle.
\item NUMBER OF REFERENCES EXCEEDS {\em number}\\
                 The text contains more text references than the
                 package can accept.
\item NO WORD SELECTION COMMANDS PROVIDED\\
                 The commands EVERY WORD or SELECT WORDS are
                 absent or mis-spelt.
\item NUMBER EXPECTED AT THIS POSITION\\
                 The previous CLOC keyword must be followed by a number.
\item A NUMBER CANNOT BE PLACED HERE\\
                 The previous CLOC keyword must {\em not} be followed by a number.
\item ZERO NUMBER NOT PERMITTED
\item SPACE NOT ALLOWED HERE
\end{itemize}

 The package makes several checks on its operation and in certain
 instances may fail with the message SYSTEM ERROR {\em number}.  Such an
 occurrence should be reported to your local advisory service.

\subsection{Warning Messages}
 These are produced when the package finds a simple mistake in a command
 not important enough to cause a fatal error.  The mistake will be
 ignored and the next command will be examined.
 The figure ~1~ is printed under the faulty position on the line, ~2~ under
the second position (and so on up to ~9~ ).
 Warning messages take the form:-
\begin{quote}
WARNING - {\em text of message}
\end{quote}
where the text is one of the following.
\begin{itemize}
\item  CHARACTER ALREADY DEFINED\\
          The *SEPARATORS, *PADDING commands etc. contain repeated
          characters.
\item  SPURIOUS CHARACTERS FOUND AND IGNORED\\
          The specification field contains characters which should
          not be present, they will be ignored.
\item  SET DESCRIPTION SPECIFIES NO WORDS\\
          The combination of word selection commands resulted in no
          words selected.
\item  WORD(S) NOT FOUND\\
          The indicated words on the *LIST OF WORDS command
          are not present in the vocabulary of the text.
\item  ABOVE ITEM TOO LONG\\
          The word printed is longer than the system can cope
          with, trailing letters have been removed.
\item  NO REFERENCE INFORMATION IN TEXT\\
          The citation option REFS was chosen but no text references
          were placed in the text.
\item  NO WORDS MATCH THIS PATTERN\\
          The current vocabulary does not contain words of this form.
\end{itemize}

 The package makes several checks on its operation and in certain instances
 may produce the message SYSTEM WARNING {\em number}. Such an occurrence should
 be reported to your local computer advisory service.

\subsection{Comment messages}
 These are produced when the system has read the text and is about to read
 the task selection commands.
\begin{itemize}
\item  TEXT FILE {\em name} SAVED ON {\em date} AT {\em time}\\
          The text has been read and stored in a permanent
          file to be later used by the GET TEXT command.
\item  TEXT FILE {\em name} ACCESSED.  (SAVED ON {\em date} AT {\em time})\\
          The GET TEXT command has accessed this file.
\item  The TEXT CONTAINS:\\
          {\em integer}1 RUNNING WORDS\\
          {\em integer}2 DISTINCT WORDS\\
          AND THE MAXIMUM WORD LENGTH IS {\em integer}3 CHARACTERS.
\end{itemize}
  The text under analysis is {\em integer}1 words in length, and the vocabulary
used contains {\em integer}2 different words. The longest word or words contain
 {\em integer}3 characters.

\section{References}
\begin{itemize}
\item  `The RUNCLOC macro', Computer Centre Users Manual,
 University of Birmingham
\item  `CLOC - An Applications Package in ALGOL 68R'
 Presented to `Applications of ALGOL 68' Conference,
 April 1975, University of Liverpool.
\item  `English Lexical Studies', J.  McH.  Sinclair, S. Jones and
 R.  Daley.
\item  `The COCOA Manual',D.B. Russell,ATLAS Computer Laboratory.
\item  `Statistical Package for the Social Sciences (SPSS)',
 N.H.  Nie, D.H.  Bent, and C.H.  Hull,
 Publ.  McGraw-Hill, New York, 1970.
\item  `CLOC: A Collocation Package', ALLC Bulletin,
       Vol. 5, No. 2, 1977.
\item  `RATS: A Middle-level Text Utility System': Smith;
    Computers and the Humanities, Vol. 6, P.277.
\item  `JEUDEMO: A Text-Handling System': Bratley, Lusigaan, and Ouellette,
     Computers in the Humanities. Pub. Edinburgh University Press.
\item  `Computer Analysis of Natural Language': Reed 1973, Birmingham
     University Computer Centre, internal report 1973.
\item  `CLOC User Guide', A.Reed,1975,Computer Centre,University of Birmingham.
\item  `OXEYE: A Text Processing Package for the 1906A', L. Burnard, 1976,
     Oxford University Computing Service.
\item  `CLOC: A General-purpose concordance and collocations generator'; A. Reed,
     J. L. Schonfelder, 1979, Aston University.
\item  `OCP: Oxford Concordance Program', S. Hockey and I Marriott, October 1980,
     Oxford University Computing Service.
\item  `Anatomy of a Text Analysis Package', A. Reed, Computer Lang., Vol. 9,
     No. 2, pp 89-96, 1984.
\end{itemize}

\section{Glossary}
\begin{description}
\item[LETTER]  One of an arbitrary collection of graphic signs, used to
               construct words.
\item[SEPARATOR] This is composed of:-
\begin{enumerate}
\item A graphic sign which is not a LETTER.
\item An arbitrary sequence of 1 above.
\end{enumerate}
\item[WORD]  An arbitrary sequence of LETTERs, generally contiguous,but may
             contain graphic signs which are totally ignored during reading.
\item[NODE]  A particular word about which a concordance can be printed or a
        collocation analysis performed.
\item[SPAN]  The context of words, surrounding a NODE, which is used during
        collocation analysis.
\item[COLLOCATE]  one of the words of context in a SPAN.
\end{description}

\section{CLOC Global Syntax Rules}
  The following `railroad' diagram describes the syntax rules for CLOC
  control statements. Follow the arrows from top to bottom and you will pass
  through all compulsory commands.  A diversion of route indicates
  optional commands.  A choice of route indicates a choice of commands
  at that position.
\begin{verbatim}
                                                !
               ...............     .............!..........
               !INPUT DETAILS!<----!                      !
               ~~~~~~~!~~~~~~~     !                      !
                      `----------->!                      !
                         ..........!..........       .....!....
                         !ITEMIZE USING  CLOC!       !GET TEXT!
                         ~~~~~~~~~~!~~~~~~~~~~       ~~~~~!~~~~
                              .....!....                  !
                              !*LETTERS!                  !
                              ~~~~~!~~~~                  !
.--------------------------------->!                      !
!  ..........                      !                      !
!--!*PADDING!<-------.             !                      !
!  ~~~~~~~~~~        !             !                      !
!  ...........       !             !                      !
!--!*DEFERRED!<------!             !                      !
!  ~~~~~~~~~~~       !             !                      !
!  .............     !<------------!                      !
!--!*SEPARATORS!<----!             !                      !
!  ~~~~~~~~~~~~~     !             !                      !
!  ................  !             !                      !
!--!*READ AS SPACE!<-!             !                      !
!  ~~~~~~~~~~~~~~~~  !             !                      !
!  .........         !             !                      !
`--!*IGNORE!<--------'             !                      !
   ~~~~~~~~~      ...........      !                      !
                  !SAVE TEXT!<-----!                      !
                  ~~~~~!~~~~~      !                      !
                       `---------->!<---------------------'
                ................   !
                !OUTPUT DETAILS!<--!
                ~~~~~~~!~~~~~~~~   !
                       `---------->!
.--------------------------------->!<--------------------------------------.-.
!      .---------.---------.---<---+-->--.--------.-------------.          ! !
!  ....!.... ....!.... ....!....   !  ...!.. .....!..... .......!.......   ! !
!  !NEWLINE! !NEWPAGE! !MESSAGE!   !  !NOTE! !WRITETEXT! !CO-OCCURRENCE!   ! !
!  ~~~~!~~~~ ~~~~!~~~~ ~~~~!~~~~   !  ~~~!~~ ~~~~~!~~~~~ ~~~~~~~!~~~~~~~   ! !
`<-----'---------'---------'       !     !<-------'             !<---------! !
                                   !     !      .<--------.-----+--->.     ! !
                                   !     !      !     ....!....  ....!.... ! !
                                   !     !      !     !*PHRASE!  !*SERIES! ! !
                                   !     !      !     ~~~~!~~~~  ~~~~!~~~~ ! !
                                   !     !      !         `----------`---->' !
                                   !     !      `------>.--------->.       ! !
                                   !     !         .....!....  ....!....   ! !
                                   !     !         !*PATTERN!  !*CHOICE!   ! !
                                   !     !         ~~~~~!~~~~  ~~~~!~~~~   ! !
                                   !     !              `----------`------>' !
                                   !     !                                   !
                                   !     `---------------------------------->'
\end{verbatim}
\newpage
\begin{verbatim}
                                   !
.--------------------------------->!
!        .-------------------------+----------------------.
!  ......!.......            ......!.....             ....!...
!  !SELECT WORDS!            !EVERY WORD!             !FINISH!
!  ~~~~~~~!~~~~~~            ~~~~~~!~~~~~             ~~~~~~~~
!  .......!.........               !
!  !Set description!-------------->!
!  ~~~~~~~~~~~~~~~~~               !
!        .------------------------>!<-----------------------.
!        !      ...........        !       ...........      !
!        !      !EXCLUDING!<-------!------>!INCLUDING!      !
!        !      ~~~~~!~~~~~        !       ~~~~~!~~~~~      !
!        !   ........!........     !    ........!........   !
!        `<--!Set description!     !    !Set description!-->'
!            ~~~~~~~~~~~~~~~~~     !    ~~~~~~~~~~~~~~~~~
!--------------------------------->!
!     .-----------.------.---------+-----------.--------------.---------------.
!  ...!......     ! .....!......         ......!....... ......!........       !
!<-!WORDLIST!     ! !STATISTICS!         !COLLOCATIONS! !CO-OCCURRENCE!       !
!  ~~~~~~~~~~     ! ~~~~!~~~~~~~         ~~~~~~~~~!~~~~ ~~~~~~~~~!~~~~~       !
!  .......        !     !  ..........  .......    !              !<---------. !
!<-!INDEX!<-------!     !->!*PROFILE!  !*SPAN!<---!   .----.<----+--->.     ! !
!  ~~~~~~~        !     !  ~~~~~!~~~~  ~~~!~~~    !   ! ...!..... ....!.... ! !
!  .............  !     !<------'         `------>!   ! !*PHRASE! !*SERIES! ! !
!<-!CONCORDANCE!<-'     !           ............  !   ! ~~~~!~~~~ ~~~~!~~~~ ! !
!  ~~~~~~~~~~~~~        !           !*FREQUENCY!<-!   !     `---------`---->' !
!<----------------------'           ~~~~~~!~~~~~  !   !  ..........         ! !
!                                         `------>!   !->!*PATTERN!-------->' !
!                     .<--------------------------'   !  ~~~~~~~~~~         ! !
! .................   !   .................           !  .........          ! !
! !EVERY COLLOCATE!<--+-->!SELECTCOLLOCATE!           !->!*CHOICE!-------->'  !
! ~~~~~~~!~~~~~~~~~   !   ~~~~~~~~!~~~~~~~~           !  ~~~~~~~~~            !
!        !            !   ........!........           !  ...........          !
!        !            !   !Set description!           !<-!WRITETEXT!<---------!
!        !            !   ~~~~~~~~!~~~~~~~~           !  ~~~~~~~~~~~          !
!        `----------->!<----------'                   !  .........            !
! .------------------>!<--------------------.         !<-!NEWLINE!<-----------!
! !    ...........    !    ...........      !         !  ~~~~~~~~~            !
! !    !REJECTING!<---!--->!ACCEPTING!      !         !  .........            !
! !    ~~~~~!~~~~~    !    ~~~~~!~~~~       !         !<-!NEWPAGE!<-----------!
! ! ........!........ !  .......!.........  !         !  ~~~~~~~~~            !
! `-!Set description! !  !Set description!->'         !  ......               !
!   ~~~~~~~~~~~~~~~~~ !  ~~~~~~~~~~~~~~~~~            !<-!NOTE!<--------------!
!                     !                               !  ~~~~~~               !
!                     !                               !  .........            !
!                     !                               !<-!MESSAGE!<-----------'
`---------------------'<------------------------------'  ~~~~~~~~~
                                .<----------------------.
                                !    ............       !
                                ! .->!*FREQUENCY!------'!
Where                           ! !  ~~~~~~~~~~~~       !
                                ! !  ................   !
            Set description  ---`-!->!*LIST OF WORDS!--'!
                                  !  ~~~~~~~~~~~~~~~~   !
                                  !  ..........         !
                                  `->!*PATTERN!-------->'
                                     ~~~~~~~~~~
\end{verbatim}
\end{document}
